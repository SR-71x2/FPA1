% !TEX root =  master.tex
\chapter{Das PCR Verfahren}
\section{Übersicht der Covid19-Teststrategie}
\textbf{Rapid-Antigen-Schnelltest} eignen sich durch geringe Kosten und schnelle Ergebnisse für eine Massentestung auf Bevölkerungsebene.\footnote{Quelle tägliche Kosten Test}
Ihre Qualität unterscheidet sich allerdings deutlich zwischen den Herstellern.\footnote{Zerforschung / Schnelltesttest}
Die Quote von falsch-positiven Ergebnissen ist sehr niedrig, was für eine Massentestung auf Bevölkerungsebene essentiell ist.\footnote{Quelle Bayessches Theorem}
Die Probe ist hierbei nach Entnahme nur 60min stabil,\footnote{Quelle Schnelltest nur 60min stabil}
sodass die Auswertung vor Ort erfolgen muss.
Ein positiver Schnelltest ist Voraussetzung für die Teilnahme am PCR-Verfahren.\footnote{Quelle Verordnung}

Das \textbf{PCR-Verfahren} ist seit vielen Jahren der Standard in der Forensik\footnote{Quelle Forensik PCR}
und im Nachweis von Viruserkrankungen. Es bietet eine hohe Erkennungsrate und ist für geschultes Personal relativ einfach durchführbar. Notwendig sind allerdings spezielle Geräte, weshalb die Tests üblicherweise nicht vor Ort sondern in Laboren durchgeführt werden. Die eigentliche Testzeit von 4-5 Stunden wird hierdurch um den Transportweg der Proben verlängert.

\textbf{Cartridge-Based-NAAT} ist ein Verfahren, welches mit zu PCR vergleichbarer Präzision bereits 60 Minuten ein Ergebnis liefert. Es nutzt Einweg-Container für die Proben jedes Patienten, welche durch ein vollautomatisches Diagnosegerät verarbeitet werden. Der hohe Automatisierungsgrad soll die Reduzierung von Kosten und Fehlern ermöglichen. Die Methode wurde wenige Jahre vor der Pandemie gegen Tuberkulose entwickelt und zwischenzeitlich auf den neuen Virustyp angepasst.
\footnote{60min bis ergebnis, Sens 80-80, Spez 90-95 Quelle Auskommentiert} %https://factly.in/explainer-what-are-the-different-types-of-tests-being-used-in-india-for-covid-19-detection/  %https://www.youtube.com/watch?v=FJFXYDP8N7M

\textbf{IgG Antigen Tests} messen durch eine Blutentnahme den Spiegel der neutralisierenden Antikörper.
Das Verfahren wird zur Erkennung einer vergangenen Infektion und zur Kontrolle der Impfwirksamkeit eingesetzt.
Zur Diagnostik einer akuten Infektion ist es nicht geeignet, weshalb es für diese Arbeit keine Relevanz hat.
\footnote{Quelle IgG Antikörper / Paper von Lenz-Website}

\cleardoublepage

\section{Funktionsweise des PCR-Verfahrens}
Das PCR-Verfahren (ploymerase-chain-reaction) hat zum Ziel, das Vorhandensein einer Gensequenz in einer Probe nachzuweisen.
Im Zusammenhang mit der Pandemie soll überprüft werden, ob in der Speichelprobe eines Patienten die Gensequenz von SARS-CoV2 vorhanden ist, was auf eine akute Infektion hindeuten würde.\footnote{Quelle Interpretation}

Die Probe wird hierfür nach der Entnahme zunächst in einer Flüssigkeit gelöst, welche Proteine und Fette auflöst.\footnote{Quelle Trägerflüssigkeit}
Hierdurch wird in einer positiven Probe die Hülle des Virus aufgelöst, sodass dessen RNA frei in der Flüssigkeit treibt.

Die Flüssigkeit wird dann in mehreren Zyklen erhitzt und abgekühlt.
Durch das Erhitzen verliert der DNA-Doppelstrang seine Wasserstoffbrückenbindung und löst sich zu zwei RNA-Einzelsträngen.\footnote{Quelle Auswirkung erhitzen}
Diese liegen anschließend einzeln vor und können nach dem Abkühlen von der RNA-Polymerase wieder zu einem Doppelstrang vervollständigt werden.\footnote{Quelle Polymerase}
Die DNA-Menge wird hierdurch in jedem Zyklus verdoppelt.
Ziel dieser Polymerase-Kettenreaktion ist es, die Ziel-DNA durch genug Zyklen so lange zu vermehren, bis eine messbare Menge vorliegt.
Die Anzahl der Zyklen wird dabei als ct-Wert angegeben.
Eine gängige Zyklenanzahl sind XX Verdopplungsschritte, wobei nicht immer eine exakte Verdopplung stattfindet.\footnote{Quelle ct Wert}

Das gesamte Verfahren benötigt im Falle von SARS-CoV2 üblicherweise vier bis fünf Stunden bis genug Genmaterial für den Nachweis vorliegt.\footnote{Quelle Dauer}
Die Erkennungsrate (Sensitivität) des Verfahrens liegt bei XX Prozent.\footnote{Quelle Sensitivität PCR}

\cleardoublepage

\section{Möglichkeiten und Grenzen zum Pooling bei PCR}
Das PCR-Verfahren erlaubt grundsätzlich ein Pooling von Proben.
Die Proben mehrerer Patienten können somit zu einem Pool zusammengefasst und gemeinsam getestet werden.
Die Verwässerung der Probe ist bis zu einem gewissen Punkt unproblematisch für den Nachweis, da das PCR-Verfahren darauf ausgelegt ist geringe DNA-Mengen zu einer nachweisbaren Menge zu vermehren.
Es ist zu beachten, dass hierdurch mehr Verdopplungsschritte notwendig sind um dieselbe Virenmenge in der Probe zu erhalten.
Beim Pooling von 16 Personen liegt beispielsweise eine um $2^{4}$ niedrigere Virenlast vor.
Deshalb müssen 4 weitere Zyklen eingeplant werden.
\footnote{Quelle 1 Pooling Verwässerung}
\footnote{Quelle 2 Pooling Verwässerung}

\subsubsection{Unklare Ergebnisse und Nachtestung}
Durch Pooling besteht - abhängig vom Verfahren - das Risiko, dass die Ergebnisse nicht für alle Testpersonen eindeutig interpretiert werden können.
Bei vielen Pooling-Verfahren ergibt sich hierdurch die Notwendigkeit einer Nachtestung.
Ob dies der Fall ist und welche Quote der Testpersonen nachuntersucht werden muss, ist abhängig vom gewählten Verfahren.

Durch die erneute Testung geht Zeit verloren, bevor für alle Testpersonen das Ergebnis fest steht.
Die Proben müssen zudem ausreichend umfangreich sein, um genug Substanz für mehrere Testungen zu enthalten.
Beim Pooling des ersten Durchlaufs muss darauf geachtet werden, die Proben untereinander nicht zu kontaminieren.

\subsubsection{Prävalenz}
Die Prävalenz ist die Quote, mit welcher eine Krankheit in einer Stichprobe vorkommt.
Sie ist ähnlich der derzeit allgemein bekannteren Inzidenz, welche sich auf die Gesamtbevölkerung bezieht.
Bei einer anlasslosen, repräsentativen Testung der Bevölkerung kann die Prävalenz eines Tests gleich der Inzidenz sein.

Bei einer anlassbezogenen Testung werden allerdings meist deutlich höhere Prävalenzen beobachtet.

\footnote{Leon Gordis S37}
\footnote{Beispiel Zeitungsbericht 70 Prozent Prävalenz}


\subsubsection{Verhinderung von Kontamination}
Die komplette Matrix sollte vor dem Pooling einmal dupliziert werden. 
Die für den aktuellen Test notwendigen Proben werden hierbei entnommen und im Duplikat gepoolt.
Für diesen Duplikationsschritt gibt es spezialisierte Laborgeräte, sodass dies in einem Arbeitsschritt für alle Proben durchgeführt werden kann.

%Doppelt
Hierfür muss zu beginn genug Probenmaterial bereit stehen und dieses darf nicht bei der Kombination kontaminiert werden.
Es empfiehlt sich, die Testmatrix zu beginn einmal zu klonen, um in der Originalmatrix ohne kontamination einzeln nachtesten zu können.

\subsubsection{Mögliche Poolgrößen und Erkennungsrate}
Um ein zuverlässiges Ergebnis zu liefern, dürfen die Proben nicht zu stark verwässert werden.
Hierbei wird empfohlen, maximal 20 Personen in einem Pool zu kombinieren. \footnote{Vieweger v1}
Die Testgruppe kann je nach Verfahren größer sein, solange kein Pool mehr als 20 Personen enthält.
Diese Poolgröße liegt laut Viehweger "comfortable above the detection rate"\footnote{Vieweger v1}

Eine höhere Verdünnung ist zulasten der Erkennungsrate problemlos möglich.
Abgewogen werden muss hierbei die Priorisierung zwischen Präzision und Kostenersparnis.

\subsubsection{Verzögerung des Testergebnisses}
Durch mehrere Sequenzielle Tests verzögert sich das Testergebnis.
Dies kann abhängig von der Situation nicht akzeptabel sein.
Ein Zwischenergebnis "Verdachtsfall" könnte übermittelt werden, aber würde möglicherweise die Patienten verunsichern.

