% !TEX root =  master.tex
\chapter{Beispiel-Kapitel: Gebrauchsanleitung \LaTeX}

\nocite{*}

In diesem Kapitel werden die Grundlagen von \LaTeX\index{LaTeX@\LaTeX} vorgestellt.

\section{Übersicht über die Vorlage}
Die Vorlage wurde im UTF-8 Encoding erstellt. Sollten daher z.\,B. Umlaute in Ihrem \LaTeX-Editor nicht korrekt dargestellt werden, überprüfen Sie bitte die Encoding-Ein\-stel\-lun\-gen des Editors. In seltenen Fällen müssen Sie die Vorlage danach noch einmal neu in den Editor einbinden. 
Die Vorlage beinhaltet die folgenden, in Tabelle \vref{tab:dateien} aufgelisteten Dateien: 
\begin{table}[h!]
	\centering
\begin{tabular}{lp{10cm}}
	\textbf{Dateiname} & \textbf{Beschreibung}\\\toprule
	\texttt{master.tex} & Die Hauptdatei. Alle anderen Dateien werden von dieser Datei eingezogen. \\
	\texttt{abstract.tex} & Die Kurzfassung der Arbeit. \\	
	\texttt{config.tex} & Konfigurationseinstellungen 	 der einzelnen Pakete\\
	\texttt{acronyms.tex} & Definition von Abkürzungen. \\
	\texttt{titlepage.tex} & Titelseite der Arbeit. \textbf{Bitte Anpassen!}\\
	\texttt{anleitung.tex} & Diese Anleitung\\ 
	\texttt{bibliography.bib}&  Die Literaturdatenbank -- hier können Sie die verwendete Literatur einpflegen.\\
	\texttt{ewerkl.tex} & Ehrenwörtliche Erklärung. \textbf{Bitte Anpassen!}\\
	\texttt{appendix.tex} & Anhang bzw. Anhänge \\\bottomrule
\end{tabular}
\caption{\label{tab:dateien}Übersicht über die Dateien der Vorlage}
\end{table}

Es werden -- unter anderem -- die folgenden Zusatzpakete von dieser Vorlage eingezogen und sollten daher in aktuellen Versionen installiert sein: 
\begin{itemize}
	\item\texttt{KOMA-Script} bzw. die Dokumentenklasse \texttt{scrreprt}
	\item\texttt{hyperref} für PDF-Informationen und Links 
	\item \texttt{babel} für länderspezifische Einstellungen
	\item \texttt{csquotes} für sprachabhängige Anführungszeichen (Befehl: \texttt{\textbackslash enquote})
	\item \texttt{acronym} für das Erstellen des Abkürzungsverzeichnisses 
	\item \texttt{booktabs} für das typografisch schöne Setzen von Tabellen 
	\item \texttt{varioref} für einfaches Referenzieren 
	\item \texttt{listings} für schöne Quelltexte
	\item \texttt{algorithm} für schöne Algorithmen
	\item \texttt{bibltatex} und \texttt{biber} für die Erstellung des Literaturverzeichnisses.
\end{itemize}
Alle Konfigurationen dieser Vorlage\index{Vorlage} können in der Datei \texttt{config.tex} eingesehen und ggf. verändert werden. Bitte schauen Sie sich die entsprechenden Dokumentationen 
der Pakete an (\url{https://www.ctan.org}), um deren Verwendung und Möglichkeiten jenseits der hier gezeigten Beispiele zu erlernen.


\section{Übersetzung von \LaTeX-Dateien}
Die Übersetzung von \LaTeX-Dateien erfolgt in mehreren Schritten und unter der Zuhilfenahme unterschiedlicher Programme. Das 
Hauptdokument (hier die Datei \texttt{master.tex}) wird mittels \texttt{pdflatex} zu einem PDF übersetzt. 
Ggf. ist eine mehrfache Übersetzung notwendig, um z.\,B. das Inhaltsverzeichnis korrekt darzustellen. 

Für die Einbindung des Literaturverzeichnisses\index{Literaturverzeichnis} wird nicht mehr das ältere \texttt{bibtex}, sondern das neuere \texttt{biber} in Kombination mit \texttt{biblatex} verwendet. Bitte stellen Sie Ihren \LaTeX-Editor so ein, dass die Verwendung von Biber beim Übersetzungsprozess erfolgt. 

\section{Verwendung von Akronymen}
Akronyme\index{Akronym} müssen in der Datei \texttt{acronyms.tex} definiert werden (schauen Sie sich hierzu bitte die entsprechende 
Paket-Dokumentation an!). Ein definiertes Akronym kann dann mit dem Befehl \texttt{\textbackslash ac} verwenden, so wird 
z.\,B. \texttt{\textbackslash ac\{DHBW\}} zu \ac{DHBW}. Im weiteren Verlauf wird das Acronym dann nur noch in der Kurzform 
dargestellt: \ac{DHBW}. Die Aufnahme eines verwendeten Akronyms in das Abkürzungsverzeichnis erfolgt automatisch 
\autocite[Vgl.][S. 77ff]{TestOnlineQuelle}, \autocite[Vgl.][S. 42]{ME12}. 

\section{Zitieren von Quellen}
Mit dem Befehl \texttt{\textbackslash cite} kann zitiert werden. Z.\,B. so: \cite[Vgl.][S.~18ff]{ME12} oder Vgl.~\cite[S.~18ff]{ME12} 
oder \cite[S.~18ff]{ME12} oder \cite{ME12}. Sollen mehrere Referenzen auf einmal gesetzt werden, können Sie dies mit dem 
Befehl \texttt{\textbackslash cites} oder zum Teil wieder mit 
\texttt{\textbackslash cite} erreichen. Z.\,B. so: \cites[Vgl.][S. 10]{ME12}[Vgl.][S. 100]{TD15}  
oder Vgl.~\cite{ME12, TD15} oder oder \cite{ME12, TD15}. Die Übernahme der Quellen in das Literaturverzeichnis erfolgt automatisch. 
Ein Beispiel für eine Online-Quelle ist ebenfalls enthalten \cite{TestOnlineQuelle}.

Wird \texttt{cite} oder \texttt{cite} konsequent verwendet, kann in der Datei \texttt{config.tex} der Zitierstil\index{Zitierstil} umgeschaltet 
werden, ohne dass im Text Veränderungen vorgenommen werden müssen. Vorkonfigurierte Stile sind Numerisch (numeric), 
Alphabetisch (alphabetic), IEEE (ieee), Harvard (apa), Chicago (authoryear), etc.~entweder im Text (inline) oder als Fußnoten 
(footnote). Im vorliegenden Text wird der Stil (\indextype)/(\position) verwendet.

Auch mit dem Befehl \texttt{\textbackslash autocite} kann zitiert werden. Z.\,B. so: \autocite[Vgl.][S.~18ff]{ME12}
oder Vgl.~\autocite[S.~18ff]{ME12} oder \autocite[S.~18ff]{ME12} oder \autocite{ME12}{.} Sollen mehrere Referenzen auf einmal 
gesetzt werden, können Sie dies mit dem Befehl \texttt{\textbackslash autocites} erreichen. Z.\,B. so:
\autocites[Vgl.][S. 10]{ME12}[][S. 100]{TD15}. Wird \texttt{autocite} konsequent verwendet, kann in der Datei 
\texttt{config.tex} der Zitierstil umgeschaltet werden, ohne dass im Text Veränderungen vorgenommen werden müssen. 

Soll einer Abbildung eine Quellenangabe zugefügt werden, bietet es sich an, diese direkt in der jeweiligen Abbildungsbeschriftung zu hinterlegen. Hierfür kann der Befehl \texttt{\textbackslash cite} verwendet werden, um eine ungewollte Fußnote zu vermeiden. Ein Beispiel ist in Abbildung 
\vref{fig:test} zu sehen. 

\section{Text in Anführungszeichen}
Soll ein Text in Anführungszeichen gesetzt werden, kann dies über den Befehl \texttt{\textbackslash enquote} \enquote{so erreicht werden}. Die Anführungszeichen ändern sich automatisch auf die 
jeweiligen Länderspezifika, wenn die Spracheinstellung des \texttt{babel}-Pakets geändert wird. Voreinstellung ist die deutsche Verwendung von 
Anführungszeichen.

\section{Verwendung eines Index}
Wenn Sie einen Index\index{Index} oder Stichwortverzeichnis\index{Stichwortverzeichnis} erstellen wollen, de-kommentieren Sie 
den Befehl "'\texttt{\textbackslash makeindex}"' in der \LaTeX-Pr\"aambel\index{Pr\"aambel}. Um Eintr\"age Ihres Textes in den Index aufzunehmen, 
verwenden Sie an der entsprechenden Stelle im Text den Befehl "'\texttt{\textbackslash index\{<Eintrag>\}}"'. Um einen Index zu erstellen, 
verwenden Sie den Befehl \texttt{makeindex.exe}, der ggf.~auch in Ihrem \LaTeX-Editor als Shortcut enthalten ist. 
Danach m\"ussen Sie Ihre \LaTeX-Datei erneut kompilieren. Der Index wird am Ende Ihres Dokuments eingefügt.

\section{Beispiele}
\lipsum[1]

\subsection{Unterabschnitte}
Es\index{Unterabschnitte} gibt neben \texttt{\textbackslash chapter} auch noch  \texttt{\textbackslash section}, \texttt{\textbackslash subsection}, \texttt{\textbackslash subsubsection} etc. Eine zu starke Untergliederung des Textes sollte jedoch vermieden werden (z.\,B. ein Abschnitt 3.4.2.5.3). 

\subsection{Tabellen und Abbildungen}
Tabellen\index{Tabelle} und Abbildungen\index{Abbildung} sind 
sogenannte \textit{Floating Objects}, d.\,h. \LaTeX\ setzt diese Objekte an Positionen, die satztechnisch geeignet sind. Daher kann es 
vorkommen, dass Tabellen oder Abbildungen auf einer anderen Seite erscheinen, die dann referenziert werden müssen. Hier ein Beispiel dafür: 

In Tabelle \vref{tab:tabelle1} ist eine Tabelle abgebildet, die mit dem Befehl \texttt{\textbackslash vref} referenziert wurde. 
Gleiches kann man auch mit Abbildungen machen, wie z.\,B. mit der Abbildung \vref{fig:test}. \LaTeX~ kümmert sich darum, 
wo die Abbildungen gesetzt werden und passt den Text der Referenz entsprechend an. Soll nur die Nummerierung in den Text geschrieben werden, 
dann kann auch der Befehl \texttt{\textbackslash ref} verwendet werden. Abbildungen sollten -- falls möglich -- als Vektor-PDF eingebunden 
werden, da die diese dann beliebig skalieren können.

\lipsum[1]
\begin{table}
	\centering
	\begin{tabular}{p{3cm}crl}
		\textbf{Spalte 1} & \textbf{Spalte 2} & \textbf{Spalte 3} & \textbf{Spalte 4}\\\toprule
		Zeile 1 Spalte 1 &  Zeile 1 Spalte 2 & Zeile 1 Spalte 3 & Zeile 1 Spalte 4\\
		Zeile 2 Spalte 2 &  Zeile 2 Spalte 2 & Zeile 2 Spalte 3 & Zeile 2 Spalte 4\\\midrule
		Zeile 3 Spalte 1 &  Zeile 3 Spalte 2 & Zeile 3 Spalte 3 & Zeile 3 Spalte 4\\
		Zeile 4 Spalte 1 &  Zeile 4 Spalte 2 & Zeile 4 Spalte 3 & Zeile 4 Spalte 4\\\bottomrule
	\end{tabular}
	\caption[Testtabelle]{\label{tab:tabelle1}Testtabelle}
\end{table}

\begin{figure}
	\centering 
	\includegraphics[scale=0.16]{\imagedir/CRISP-DM.png}
	\captionsetup{format=hang}
	\caption[Optionaler Kurztitel für das Abbildunggsverzeichnis]{\label{fig:test}Demo-Abbildung, um zu verdeutlichen, wie gleitende Objekte 
		gesetzt werden und wie entsprechend die Quelle zitiert wird. \\Quelle: \cite[][S. 223]{TD15}}
\end{figure}
	
\subsection{Mathematische Formeln}
Auch\index{mathematisch!Formel} mathematische Ausdrücke\index{mathematisch!Ausdruck} können mit \LaTeX~ sehr gut gesetzt werden, wie man anhand der 
Gleichungen\index{mathematisch!Gleichung} \vref{eqn:e1} und \vref{eqn:e2} sehen kann -- konsultieren Sie hierzu bitte entsprechende Dokumentationen, 
die Online zur Verfügung stehen.
\begin{equation}
\left|{1\over N}\sum_{n=1}^N \gamma(u_n)-{1\over 2\pi}\int_0^{2\pi}\gamma(t){\rm d}t\right| \le {\varepsilon\over 3}.\\
\label{eqn:e1}
\end{equation}

\begin{equation}
f(x)=x^2
\label{eqn:e2}
\end{equation}

\subsection{Algorithmen}
Algorithmen\index{Algorithmus} können als Pseudocodes dargestellt und referenziert werden, wie z.\,B. in Algorithmus \vref{alg:euclid} -- 
sogar bis auf Zeilennummern (siehe die \texttt{while}-Anweisung in Zeile \vref{alg:euclid:while}). Schauen Sie sich hierzu bitte das 
Paket \texttt{algorithmicx} an.

\begin{algorithm}[H]
\begin{algorithmic}[1]
\Procedure{Euclid}{$a,b$}\Comment{The g.c.d. of a and b}
   \State $r\gets a\bmod b$
   \While{$r\not=0$}\Comment{We have the answer if r is 0} \label{alg:euclid:while}
      \State $a\gets b$
      \State $b\gets r$
      \State $r\gets a\bmod b$
   \EndWhile\label{euclidendwhile}
   \State \textbf{return} $b$\Comment{The gcd is b}
\EndProcedure
\end{algorithmic}
\caption{Euklidischer Algorithmus}\label{alg:euclid}
\end{algorithm}

Im obigen Beispiel wird der Euklidische Algorithmus in Pseudocode dargestellt. 
