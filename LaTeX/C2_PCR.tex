% !TEX root =  master.tex
\chapter{Das PCR Verfahren}
\section{Funktionsweise des PCR-Verfahrens}
Ct werte und stuff

\cleardoublepage

\section{Möglichkeiten und Grenzen zum Pooling bei PCR}
\subsubsection{Unklare Ergebnisse und Nachtestung}
Durch Pooling besteht - abhängig vom Verfahren - das Risiko, dass die Ergebnisse nicht für alle Testpersonen eindeutig interpretiert werden können.
Bei vielen Pooling-Verfahren ergibt sich hierdurch die Notwendigkeit einer Nachtestung.
Ob dies der Fall ist und welche Quote der Testpersonen nachuntersucht werden muss, ist abhängig vom gewählten Verfahren.

Durch die erneute Testung geht Zeit verloren, bevor für alle Testpersonen das Ergebnis fest steht.
Die Proben müssen zudem ausreichend umfangreich sein, um genug Substanz für mehrere Testungen zu enthalten.
Beim Pooling des ersten Durchlaufs muss darauf geachtet werden, die Proben untereinander nicht zu kontaminieren.

\subsubsection{Verhinderung von Kontamination}
Die komplette Matrix sollte vor dem Pooling einmal dupliziert werden. 
Die für den aktuellen Test notwendigen Proben werden hierbei entnommen und im Duplikat gepoolt.
Für diesen Duplikationsschritt gibt es spezialisierte Laborgeräte, sodass dies in einem Arbeitsschritt für alle Proben durchgeführt werden kann.

\subsubsection{Mögliche Poolgrößen und Erkennungsrate}
Um ein zuverlässiges Ergebnis zu liefern, dürfen die Proben nicht zu stark verwässert werden.
Hierbei wird empfohlen, maximal 20 Personen in einem Pool zu kombinieren. \footnote{Vieweger v1}
Die Testgruppe kann je nach Verfahren größer sein, solange kein Pool mehr als 20 Personen enthält.
Diese Poolgröße liegt laut Viehweger "comfortable above the detection rate"\footnote{Vieweger v1}

Eine höhere Verdünnung ist zulasten der Erkennungsrate problemlos möglich.
Abgewogen werden muss hierbei die Priorisierung zwischen Präzision und Kostenersparnis.

\subsubsection{Ziele dieser Arbeit}
Erklärtes ziel dieser Arbeit ist es, eine alternative zu Antigen-Schnelltest aufzuzeigen.
Die Rahmenbedingungen der bisherigen Testung sind somit
\begin{itemize}
	\item Sehr schnelle Anzeige des Ergebnisses
	\item Sehr geringe Kosten - unter 1€ Materialaufwand pro Stück
	\item Die Akzeptanz einer hohen Fehlerquote
	\item Massenscreenings mit sehr geringer Prävalenz
\end{itemize}
	
Um sich hiermit messen zu können, ist es notwendig die Prioritäten der PCR-Methode ähnlich festzulegen.
Einschränkungen der Präzision sind akzeptabel, um preislich mit den ungenauen Schnelltests zu konkurrieren.
Es ist außerdem eine für PCR-Verfahren unüblich niedrige Prävalenz zu erwarten, da normalerweise Verdachtsfälle mit PCR überprüft werden.

Für die Ziele dieser Arbeit ist es somit erforderlich, eine große Poolgröße zu wählen.
Diese ermöglicht eine Stärkere Kostenreduzierung.
Die Präzision fällt hierbei allerdings unter die Schwelle dessen, was üblicherweise für PCR als akzeptabel betrachtet wird.

\subsubsection{Politische Ebene}
Politisch ist anzumerken, dass bei den Corona-Schutzmaßnahmen zwischen präzissen PCR-Tests und ungenauen Schnelltests getrennt wird.
Die Teilnahme an einigen Veranstaltungen ist somit nur mit PCR-Test zulässig.
es kann unterstellt werden, dass der Gesetzgeber hierbei kein oder nur ein schwaches Pooling eingerechnet hat.
Sollte durch die gewählte Pooling-Methode die Genauigkeit deutliche reduziert sein, sollten deshalb nur Schnelltest-Bescheinigungen an die negativ getesteten Personen ausgestellt werden.

\cleardoublepage
\cleardoublepage
\section{Methoden für PCR-Pooling}
\cleardoublepage
\cleardoublepage
