% !TEX root =  master.tex
\chapter*{Kurzfassung (Abstract)}
\addcontentsline{toc}{chapter}{Kurzfassung (Abstract)}

\subsubsection{Deutsch}
Die vorliegende Arbeit hat zum Ziel, die Effizienz der PCR-Analyse durch Pooling der Proben zu erhöhen.
Beantwortet werden soll die Frage, ob sich die Effizienz des PCR-Verfahren durch Pooling stark genug steigern lässt, um eine wirtschaftliche Alternative für das Einsatzgebiet von Schnelltests zu bieten?
Weiter wird untersucht, wie eine PCR-basierte Teststrategie im betrieblichen Umfeld realisiert werden könnte.

Erläutert wurden zunächst Methoden für die Testung und Grundlegende Konzepte des Poolings.
Es wurde die Effizienzsteigerungspotenzial durch Pooling aufgezeigt und errechnet.
Auf dieser Basis wurden PrePrints mit weitergehenden Poolingkonzepten analysiert, um zu prüfen ob diese weiteres Steigerungspotenzial bieten.

Ergebnis...

\subsubsection{Englisch} %DEEPL.com
The aim of this research paper is to increase the efficiency of PCR analysis by pooling samples.
The question to be answered is whether the efficiency of the PCR process can be increased strongly enough by pooling to offer an economical alternative for the application of rapid tests?
Further, it will be investigated how a PCR-based testing strategy could be realized in an operational environment.

Methods for testing and basic concepts of pooling were first explained.
The potential for efficiency gains through pooling was shown and calculated.
On this basis, PrePrints with more advanced pooling concepts were analyzed to see if they offer further potential for increase.