% !TEX root =  master.tex
\chapter{Einleitung}
\section{Problemstellung}
Die Covid19-Pandemie ist zum aktuellen Zeitpunkt im Februar 2022 seit 23 Monaten ausgerufen.
Mit unterschiedlichsten Maßnahmen wird versucht, die weitere Ausbreitung einzudämmen und die Kapazitäten des Gesundheitssystems in den kommenden Wintermonaten nicht zu überlasten.
Neben Impfungen und Masken zählen auch Einschränkungen des öffenlichen Lebens und die Nachverfolgung von Infektionsketten zu den ergriffenen Maßnahmen.

Ein weiterer elementarer Baustein der Pandemiestrategie ist die anlassbezogene, aber auch anlasslose Massentestung der Bevölkerung auf Infektion mit dem neuartigen Coronavirus.
Hierbei erfolgt die Massentestung üblicherweise mit Antigen-Schnelltests, während Verdachtsfälle über PCR-Tests überprüft werden.
\footnote{(Quelle Verordnung)}
 Der Nachweis einer negativen Testung ist - alternativ zu einer Immunisierung - Voraussetzung für die Teilnahmen an vielen Bereichen des öffentlichen Lebens.

Unternehmen sind aktuell verpflichtet Ihren Mitarbeitern zweimal pro Woche eine Testmöglichkeit zu bieten.
Hierfür kommen nahezu ausschließlich Antigen-Schnelltests zum Einsatz.
Diese Stückpreise schwanken hier zwischen (Betrag) Euro
\footnote{(Onlinepreis)}
und (Betrag) Euro.
\footnote{(Onlinepreis)}

Die Sensitivität der Schnelltests ist allerdings nach aktueller Auffassung nicht ausreichend.
Die Uni Würzburg testete die Sensitivität mehrerer Anbieter in XX/2021 und kam hierbei zu einer Sensitivität von 42,6 Prozent.
Die Zulassungsstudien der Schnelltests zeigen eine sehr große Bandbreite in der Qualität.
(Hersteller) erkennt selbst bei geringer Viruslast XX Prozent der Infektionen
(Hersteller) dagegen zeigt selbst bei sehr hoher Virenlast nur XX Prozent der Infizierten richtig an.
\footnote{Zerforschung Jan 2022}

Aus diesem Grund erfordert die Corona-Verordnung-BW bei einem verstärkten Infektionsgeschehen in vielen Bereichen einen PCR-Test für ungeimpfte Personen.
Die Erneute Aufnahme der Testung von geimpften Personen wurde kürzlich mit 2G-Plus für viele Veranstaltungen beschlossen.
\footnote{Verordnung / Zeitung MPK}

Die Uni Würzburg testete im XXXX 2021 die Produkte mehrerer Hersteller.
Das Ergebnis war eine durchschnittliche Sensitivität von 42,6 Prozent.
Die Entdeckungsrate ist abhängig von der Viruslast, welche wiederum von Variante, Impfstatus und Individuellem Verlauf abhängt.

Neue Varianten können Anpassungen an Testverfahren sowie bei den Impfstoffen erfordern.

\section{Zielsetzung}
Bestehende Verfahren sollen analysiert und miteinander vergleichen werden.
Im Laufe der Arbeit soll hierfür ein Kriterienkatalog entwickelt werden.

Im Laufe der Pandemie wurden von vielen Forschungsgruppen und Laboren Methoden entwickelt, um PCR-Pooling durchzuführen. Die Skalierung ist hier sehr unterschiedlich und es gibt widersprüchliche Aussagen dazu, wie robust das PCR Verfahren gegen Verwässerung der Proben ist. Einige behaupten man könne maximal 5 Personen gemeinsam testen. Andere testen 25-40 gemeinsam.

In Deutschland haben die größte Erfahrung die Blutspendedieste zu haben, da diese seit Jahrzehnten Pooling-Verfahren einsetzen um auf HIV und Hepatitis zu testen (Ärtzeblatt). Diese haben hierfür auch ein Patent angemeldet. Die Methode dieses Patents soll die Basis für den Vergleich anderer Verfahren sein.

\subsubsection{Forschungsfragen}
\textbf{Primäres Forschungsziel}\newline
Analyse existierender PCR-Pooling-Verfahren.\newline
Überprüfung dieser Methoden auf Effizienz und Robustheit.

\textbf{Sekundäres Forschungsziel}\newline 
Erarbeitung einer Referenzimplementierung für das betriebliche Umfeld.\newline

Neu Schreiben:
Ermittlung von Kostensenkungspotenzialen bei der betrieblichen Testung durch PCR-Poolingverfahren.

Ist es möglich, das PCR-Verfahren auf einer Kosten-Präzissions-Basis effizienter durchzuführen als Schnelltests.

\subsubsection{Ziele dieser Arbeit}
Erklärtes ziel dieser Arbeit ist es, eine alternative zu Antigen-Schnelltest aufzuzeigen.
Die Rahmenbedingungen der bisherigen Testung sind somit
\begin{itemize}
	\item Sehr schnelle Anzeige des Ergebnisses
	\item Sehr geringe Kosten - unter 1€ Materialaufwand pro Stück
	\item Die Akzeptanz einer hohen Fehlerquote
	\item Massenscreenings mit sehr geringer Prävalenz
\end{itemize}

Um sich hiermit messen zu können, ist es notwendig die Prioritäten der PCR-Methode ähnlich festzulegen.
Einschränkungen der Präzision sind akzeptabel, um preislich mit den ungenauen Schnelltests zu konkurrieren.
Es ist außerdem eine für PCR-Verfahren unüblich niedrige Prävalenz zu erwarten, da normalerweise Verdachtsfälle mit PCR überprüft werden.

Für die Ziele dieser Arbeit ist es somit erforderlich, eine große Poolgröße zu wählen.
Diese ermöglicht eine Stärkere Kostenreduzierung.
Die Präzision fällt hierbei allerdings unter die Schwelle dessen, was üblicherweise für PCR als akzeptabel betrachtet wird.

\subsubsection{Politische Ebene}
Politisch ist anzumerken, dass bei den Corona-Schutzmaßnahmen zwischen präzissen PCR-Tests und ungenauen Schnelltests getrennt wird.
Die Teilnahme an einigen Veranstaltungen ist somit nur mit PCR-Test zulässig.
es kann unterstellt werden, dass der Gesetzgeber hierbei kein oder nur ein schwaches Pooling eingerechnet hat.
Sollte durch die gewählte Pooling-Methode die Genauigkeit deutliche reduziert sein, sollten deshalb nur Schnelltest-Bescheinigungen an die negativ getesteten Personen ausgestellt werden.

%\subsubsection{Gang der Untersuchung}

%#########################################################
%#########################################################
\if{false}
\section{Planung aus RP}
Die Einleitung beginnt mit einer Beschreibung der Problemstellung und Zielsetzung der Arbeit.
Hierauf folgen die Forschungsfragen, welche sich am Research Proposal orientieren.
Am Ende des Einleitungskapitels wird das PCR-Verfahren im Allgemeinen vorgestellt, um eine Grundlage für die Bearbeitung der Forschungsfragen zu schaffen.

\section{Obsidian Sammlung}
\subsection{Problemstellung:}
Pandemiestatistiken:
- Die Covid19-Pandemie ist zum aktuellen Zeitpunkt im XXX 2022 seit XX Monaten ausgerufen.

Ziel der Maßnahmen
- die weitere Ausbreitung einzudämmen
- die Kapazitäten des Gesundheitssystems in den Wintermonaten nicht zu überlasten.

Maßnahmen:
- Impfungen
- Masken
- Einschränkungen des öffenlichen Lebens
- Nachverfolgung von Infektionsketten

Tests
- anlassbezogene Testung
--- Verdachtsfälle über PCR-Tests
- anlasslose Massentestung der Bevölkerung
--- Antigen-Schnelltests
- Zutrittstests
--- 2G / 3G / 2G+ / Relation Testung
- Tests in Unternehmen
--- Unternehmen sind aktuell verpflichtet ... (aktuelle Verordnung zu Tests)

Die Sensitivität der Schnelltests ist allerdings nach aktueller Auffassung nicht ausreichend. Die Uni Würzburg testete die Sensitivität mehrerer Anbieter in XX/2021 und kam hierbei zu einer Sensitivität von 42,6 Prozent. Fast 6/10 mit Covid19 Infizierte Personen werden durch den Schnelltest somit nicht erkannt.

Aus diesem Grund erfordert die Corona-Verordnung-BW bei einem verstärkten Infektionsgeschehen in vielen Bereichen einen PCR-Test für ungeimpfte Personen. Die Erneute Aufnahme der Testung von geimpften Personen wird diskutiert (Quelle). Allerdings ist die Sensitivität der Schnelltests bei Geimpften Personen aufgrund der niedrigeren Virenlast noch schlechter. (Quelle)

footnote{Die Uni Würzburg testete im XXXX 2021 die Produkte mehrerer Hersteller.
	Das Ergebnis war eine durchschnittliche Sensitivität von 42,6 Prozent.}

Bei geimpften Personen ist die Sensitivität der Schnelltests aufgrund der niedrigeren Virenlast noch schlechter.
footnote{Q: Niedrigere Virenlast Geimpfte}

Während der Erarbeitung dieses Research Proposals wurde beispielsweise eine neue Virusvariante B.1.1.529 "Omikron" in Südafrika entdeckt.
\footnote{Zeitung Omikron}
Ersten Beobachtungen legen nahe, dass Anpassungen an Testverfahren sowie bei den Impfstoffen notwendig werden könnten.
\footnote{Resistenz Omikron}

Bestehende Verfahren sollen analysiert und miteinander vergleichen werden.
Im Laufe der Arbeit soll hierfür ein Kriterienkatalog entwickelt werden.

Im Laufe der Pandemie wurden von vielen Forschungsgruppen und Laboren Methoden entwickelt, um PCR-Pooling durchzuführen. Die Skalierung ist hier sehr unterschiedlich und es gibt widersprüchliche Aussagen dazu, wie robust das PCR Verfahren gegen Verwässerung der Proben ist. Einige behaupten man könne maximal 5 Personen gemeinsam testen. Andere testen 25-40 gemeinsam.

In Deutschland haben die größte Erfahrung die Blutspendedieste zu haben, da diese seit Jahrzehnten Pooling-Verfahren einsetzen um auf HIV und Hepatitis zu testen (Ärtzeblatt). Diese haben hierfür auch ein Patent angemeldet. Die Methode dieses Patents soll die Basis für den Vergleich anderer Verfahren sein.

\subsection{Abgrenzung}
Das Verfahren bricht zusammen, wenn zu viele Personen positiv getestet werden.
Daher eignet es sich nur für anlasslose Tests.
Personen mit Symoptomen müssen auf klassische Weise verifiziert werden. Dies ist auch sicherer.
\fi