% !TEX root =  master.tex

\chapter{Reserch Proposal vom 30.11.2021}
\section{Problemstellung und Forschungsziel}
Der Testung der Bevölkerung ist eine der wichtigsten Maßnahmen zur Pandemiebekämpfung.
Zielführend wäre deshalb, hierbei die ungenauen Antigen-Schnelltests\footnote{WuerzburgStudie}
durch das präziseren PCR-Verfahren\footnote{Polymerase Chain Reaction}
zu ersetzen.
Aktuell ist dies aufgrund der hierfür erforderlichen Laborkapazitäten und hohen Kosten nicht umsetzbar.\footnote{rki-bericht 2021}

Die vorgesehene Arbeit hat zum Ziel, existierende Pooling-Verfahren zu überprüfen.
Die Proben mehrerer Patienten werden hierbei kombiniert und gemeinsam getestet.
Diese Verfahren bieten erhebliche Steigerungspotenziale für die Kapazitäten der PCR-Testungen.\footnote{Aertzeblatt}
Bei der anlasslosen Massentestung könnte hierdurch eine Qualitätssteigerung erreicht werden.
Identifiziert werden soll ein Pooling-Verfahren, welches diese Steigerung ohne signifikanten Qualitätsverlust ermöglicht.
Ziel der Arbeit ist deshalb, Pooling-Methoden nach den folgenden Anforderungen zu prüfen:
\begin{itemize}
	\setlength{\itemsep}{-8pt}
	\item Potenzial zur Erhöhung der Testkapazitäten
	\item Robustheit gegen Fehlanwendung
	\item Risiko von falschen Ergebnissen
	\item Anwendbarkeit im betrieblichen Umfeld
\end{itemize}

Während der Pandemie haben sich viele Forschungsgruppen an Pooling-Verfahren gearbeitet.\footnote{viehweger increased 2020}
Hierbei sind viele unterschiedliche Ansätze entstanden.\footnote{verwilt evaluation 2021}
Die vorliegende Arbeit möchte diese Methoden zunächst analysieren und zu Clustern ähnlicher Verfahren aggregieren.
Anschließend soll  eine Überprüfung auf Effizienz und Robustheit der unterschiedlichen Verfahren stattfinden.
Hierfür sollen eine Simulation oder eine argumentativ-deduktive Analyse eingesetzt werden.

\textbf{Primäres Forschungsziel}\newline
Analyse existierender PCR-Pooling-Verfahren.\newline
Überprüfung dieser Methoden auf Effizienz und Robustheit.

\textbf{Sekundäres Forschungsziel}\newline 
Erarbeitung einer Referenzimplementierung für das betriebliche Umfeld.\newline

\section{Aktueller Stand der Wissenschaft}
\subsection{Das PCR-Testverfahren}
Um die Grundlagen für spätere Kapitel zu legen, soll zunächst die allgemeine Funktionsweise des PCR-Verfahrens erläutert werden.
Die verfahrensüblichen Qualiätsmerkmale Sensitivität und Spezifität werden beschrieben und auf Ablauf sowie Logistik der Testungen eingegangen.

Das PCR-Verfahren existierte bereits vor der COVID-19-Pandemie.
Es wird seit 1983
\footnote{wink pcr 1994}
zur Erkennung von Viruserkrankungen eingesetzt.
Sowohl wissenschaftliche Literatur\footnote{schmidt novel 2020}
als auch praxisnahe Publikationen\footnote{wehrle weber pcr 1994}
sind verfügbar.\footnote{clewley polymerase 1995}
Forschungsrelevante Aussagen zur Wirksamkeit und Fehlerquote der Testung werden hierbei ausschließlich auf Quellen gestützt, welche die wissenschaftlichen Qualitätsstandards erfüllen.
Für betriebswirtschaftliche und ablauforganisatorische Themenbereiche wird die Einbeziehung von praxisnahe Literatur als sinnvoll erachtet.

\subsection{PCR-Pooling-Verfahren}
PCR-Pooling-Verfahren wurden bereits vor der Pandemie zur Diagnose anderen Viruserkrankungen erfolgreich eingesetzt.
\footnote{Aertzeblatt}
Hierbei werden die Proben mehrerer Patienten vermischt, um durch einen gemeinsamen Test den Aufwand zu senken.
Im Laufe der Pandemie wurden von vielen Forschungsgruppen und Laboren Methoden entwickelt, um PCR-Pooling durchzuführen.
\footnote{calabrese how 2021}
Zur Robustheit des PCR-Verfahren gegen Verwässerung der Proben und den Skalierungsmöglichkeiten gibt es widersprüchliche Aussagen.
Einige Methoden empfehlen, dass maximal fünf Personen gemeinsam getestet werden.
\footnote{schmidt novel 2020}
Andere vermischen die Proben von 25-40 Patienten.
\footnote{verwilt evaluation 2021}
Das Ärzteblatt bescheinigt den Blutspendediensten die meiste Erfahrung mit PCR-Pooling-Verfahren.
\footnote{Aertzeblatt}
Um Blutspenden auf HIV und Hepatitis zu testen, kommen Pooling-Verfahren hier seit Jahrzehnten zum Einsatz.

Ein Vergleich dieser Pooling-Studien und der erforschten Methoden wird ein Schwerpunkt dieses Kapitels.
Ein Fokus ist hierbei die Fehleranfälligkeit der Tests bei unterschiedlichen Bedingungen und Verfahren.

Die wissenschaftliche Artikel, welche den Verfahren zugrunde liegen, werden aufgrund der Tagesaktualität teilweise von Pre-Print-Servern stammen.\footnote{viehweger increased 2020}
In diesen Fällen ist noch keinen Peer-Review erfolgt, weswegen diese Quellen besonders kritisch reflektiert und mit anderen Publikationen verglichen werden müssen.\footnote{verwilt evaluation 2021}
Die Einhaltung des wissenschaftlichen Anspruchs wird durch den Vergleich der vielen Publikationen und auf Basis der zugrundeliegenden Literatur sichergestellt.
Eine Validierung und Plausibilisierung der Methoden ist ohnehin im Rahmen der primären Forschungsfrage vorgesehen.

\subsection{Methoden der Kanalcodierung}
Ein Forschungsgebiet der Informatik ist die Integritätsprüfung von Speichern und Signalübertragungen.
Die Forscher entwickeln Algorithmen, für die Erkennung und Berichtigung von Fehlern.\footnote{hamming information 1987}
Die Anforderungen sind hierbei stark abhängig vom Anwendungsfall und der zu erwartenden Fehlerverteilung.\footnote{blahut algebraic 1992}

Dieses Kapitel dient dazu, ein Verständnis für die Funktionsweise unterschiedlicher Codierungsverfahren zu vermitteln.
Anhand von Beispielen werden die Unterschiede und Eigenschaften verschiedener Verfahren aufgezeigt.
Verdeutlicht wird hierdurch, nach welchen Kriterien Codierungsverfahren bewertet und verglichen werden können.
Aus der Kanalcodierung werden Anforderungen, Konzepte und Terminologie\footnote{dankmeier codierung 1994}
für die Analyse der Pooling-Verfahren übernommen.

Die Existenz dieses Kapitels ist noch ungewiss.
Sein Zweck wäre, die theoretische Grundlage für eine argumentativ-deduktive Analyse der Pooling-Methoden zu liefern.
Sollte diese Analyse zugunsten einer Simulation entfallen, ist dieses Kapitel obsolet.\footnote{Vgl. Gliederung der Arbeit}

\section{Forschungsmethodik und Vorgehen}
\subsection{Validierung der Modelle}
Im vorherigen Kapitel wurden anhand von internationalen Studien Verfahren für das PCR-Pooling erarbeitet.
Die ermittelten Modelle sollen in diesem Kapitel durch Forschungsmethoden der Wirtschaftsinformatik validiert werden.

Bei mangelhafter Umsetzung einer Pooling-Methode, könnten die Fehlerrate der Testverfahren steigen oder Proben kontaminiert werden.
Das Ergebnis wären ein Mehraufwand durch erneute Testung oder unentdeckte Fehldiagnosen.
Ziel ist es, eine geeignete Methode und Skalierung zu finden und diese auf Robustheit gegen Fehlern zu überprüfen.
Als Forschungsmethoden für die Validierung sind eine \textbf{Simulation} oder eine \textbf{argumentativ-deduktiven Analyse} vorgesehen.
Hierbei sollen die Grenzen der beschriebenen Verfahren erforscht werden.

Getestet werden sollen beispielsweise:
\begin{itemize}
	\setlength{\itemsep}{-8pt}
	\item Unterschiedliche Infektionswahrscheinlichkeiten in der Testgruppe
	\item Clusterbildung unter den Positivfällen
	\item Falschergebnis einzelner (Teil-)Testungen
	\item Kontaminierung der Proben
	\item Fehler bei der Probenvermischung
\end{itemize}

Aus den verfügbaren Pooling-Verfahren des vorherigen Kapitels soll so ein möglichst effizientes und robustes Modell gewählt werden.
Es sollen optimale Parameter für das Modell gefunden werden, um die in der Forschungsfrage formulierten Ziele zu erfüllen.

Die Ergebnisse der Validierung werden gegebenenfalls als Anpassungen in die Modellen eingearbeitet.
Dies soll die Grundlage für die Auswahl einer effizienten Methode sein, welche im Zuge der sekundären Forschungsfrage implementiert wird.

\subsection{Implementierung im betrieblichen Umfeld}
In diesem Kapitel soll eine Referenzimplementierung der erarbeiteten Modelle in das betriebliche Umfeld erstellt werden.
Hierbei handelt es sich um die sekundäre Forschungsfrage.
Die primären Forschung soll im Falle eines Ressourcenkonflikts priorisiert werden.\footnote{Vgl. Gliederung der Arbeit}

In der Medizin sowie im betrieblichen Umfeld, funktioniert Skalierung grundsätzlich anders als in der Informatik.
In der betrieblichen Umsetzung bedeutet die Verdopplung der Personenzahl einen massiven Mehraufwand bei Logistik und Organisation.
Um die ermittelte Effizienzsteigerung in der Praxis zu erreichen, müssen die Abläufe fehlerfrei ausgeführt werden.
Eine Aufgabe der Implementierung ist es, Risiken für operative Fehler minimieren.
Erreicht werden kann dies durch betriebliche Abläufe, Dokumentation und die Reduzierung der Arbeitsschritte.
Hierfür sollen praktische Empfehlungen gegeben werden.
Zudem wird die Logistik und entstehende Kosten der Testung beschrieben.

Dieses Kapitel wird sich auf einige Standardliteratur aus den Bereichen Prozess- und Ablauforganisation stützen.
Im Schwerpunkt handelt es sich hierbei allerdings um ein induktives Kapitel mit dem Ziel, einen Ausblick auf mögliche Implementierungsstrategien zu geben.
Eine abschließende Behandlung der betrieblichen Abläufe wird im Rahmen der Forschungsarbeit nicht möglich sein.
Aufgrund der Individualität jedes Unternehmens sollen allgemeingültige Empfehlungen gegeben werden.

\subsection{Zu erwartende Eigenbeiträge}
Die wissenschaftliche Lage verändert sich im Rahmen der Corona-Pandemie nahezu täglich und es ist schwierig, hierzu einen nachhaltigen Beitrag zu leisten.
Die vorgeschlagene Arbeit orientiert sich deshalb zwar am aktuellen Bedarf der pandemischen Lage - beschränkt sich jedoch nicht auf diesen.

Weder das PCR-Testverfahren noch die Idee zum Pooling von Proben ist neu in der COVID-19-Pandemie entstanden.
Diese Verfahren sind bewährt und werden auch nach der Pandemie noch zum Einsatz kommen.
Die Forschung nutzt die Pandemie somit als Anhaltspunkt und Praxisbeispiel, in der Hoffnung einen kurzfristigen Beitrag leisten zu können.
Die Ergebnisse sollen hierbei abstrahierbar für weitere Anwendungsfälle bleiben.

\section{Gliederung der Arbeit}
%\begin{figure}[h]
%	\centering
	%\includegraphics[height=.15\textwidth]{images/Untitled Diagram.drawio}
	%\caption{Geplanter Aufbau der Arbeit\footnotemark}
%\end{figure}

\paragraph{Einleitung}
Die Einleitung beginnt mit einer Beschreibung der Problemstellung und Zielsetzung der Arbeit.
Hierauf folgen die Forschungsfragen, welche sich am Research Proposal orientieren.
Am Ende des Einleitungskapitels wird das PCR-Verfahren im Allgemeinen vorgestellt, um eine Grundlage für die Bearbeitung der Forschungsfragen zu schaffen.

\paragraph{Primäre Forschungsfrage}
Die Beantwortung der primären Forschungsfrage beginnt mit einer Aufbereitung der bisherigen Forschung zu PCR-Pooling-Verfahren.
Für die Validierung sind zwei Forschungsansätze denkbar:
\begin{itemize}
	\setlength{\itemsep}{-8pt}
	\item Qualitativer Ansatz:
	Die Disziplin der Kanalcodierung wird vorgestellt und auf Basis dieses wissenschaftlichen Frameworks die Pooling-Verfahren formalisiert.
	Hierauf erfolgt eine argumentativ-deduktive Analyse, welche die Methode durch theoretische und qualitative Ansätze überprüft.
	\item Quantitativer Ansatz:
	Die Pooling-Verfahren werden in Software nachgebaut und quantitativ anhand einer Simulation analysiert.
	Es werden unterschiedliche Grenzfälle getestet, um die Auswirkung auf das Verfahren zu beobachten.
\end{itemize}

\paragraph{Sekundäre Forschungsfrage und Ausblick}
Die sekundäre Forschungsfrage ist die Implementierung der Methode im betrieblichen Umfeld.
Der Umfang dieser Forschung wird flexibel dem Ressourcenbedarf der primären Forschungsfrage angepasst.
Die Behandlung der Implementierung ist somit als eigenes Hauptkapitel denkbar.
Alternativ erfolgt eine Kürzung als Ausblick nach dem Ergebnis.

\paragraph{Ergebnis}
Die Arbeit endet mit einem Kapitel, in welchem die Erlebnisse zusammengefasst und Handlungsempfehlungen gegeben werden.
Es wird geprüft ob das Forschungsziel erreicht wurde und ob ein Optimierungspotenzial gegenüber den bisherigen Verfahren besteht.
Abhängig vom vorherigen Kapitel folgt ein Ausblick.