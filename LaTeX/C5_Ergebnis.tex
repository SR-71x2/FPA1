% !TEX root =  master.tex
\chapter{Ergebnis}
\subsubsection{Erkenntnisse der Arbeit}
Zusammenfassend kann aus der Forschungsarbeit abgeleitet werden:

\begin{itemize}
	\item \textbf{Pooling} Das PCR-Verfahren kann genutzt werden, um mehrere Proben gemeinsam zu testen.\footnote{S. XX}
	\item \textbf{Einfache Poolingverfahren} Bereit durch einfache Poolingverfahren mit Nachtestung im Falle eines positiven Pools kann eine deutliche Effizienzsteigerung gegenüber der PCR-Einzeltestung erreicht werden.\footnote{S. XX}
	\item \textbf{Mehrdimensionale Poolingverfahren} Durch komplexere Analysen und mehrdimensionale Testgruppen kann eine / keine / Im Bereich von ... eine signifikante Effizienzsteigerung gegenüber einfachen Poolingverfahren erreicht werden.\footnote{S. XX - TODO}
	\item \textbf{Potenzial} Das Effizienzsteigerungspotenzial durch Pooling hängt direkt von der Prävalenz der Testgruppe ab. Das Potenzial ist bei geringer Prävalenz besonders hoch.\footnote{S. XX}
	\item \textbf{Testort} Vor Ort im Unternehmen sollte nur die Probenentnahme durchgeführt werden. Hintergrund hier sind die effizienteren Methoden und das geschulte Personal, welchen in Laboren zur Verfügung steht. \footnote{S. XX}
	\item \textbf{Verzögerung} PCR basierte Verfahren liefern zwangsläufig ein späteres Ergebnis als Schnelltests
	\item \textbf{Erkenntnis} 
	

	
\end{itemize}
	Die Nachtestung kann durch mehrstufiges Pooling optimiert werden.
\cleardoublepage
\subsubsection{Fazit zur Forschungsfrage}

%\section{Ausblick NAT}

%#########################################################
%#########################################################
\if{false}
\section{Planung aus RP}
Die Arbeit endet mit einem Kapitel, in welchem die Erlebnisse zusammengefasst und Handlungsempfehlungen gegeben werden.
Es wird geprüft ob das Forschungsziel erreicht wurde und ob ein Optimierungspotenzial gegenüber den bisherigen Verfahren besteht.
Abhängig vom vorherigen Kapitel folgt ein Ausblick.

\section{Obsidian Notizen}
Die Arbeit endet mit einem Kapitel, in welchem die Erlebnisse zusammengefasst und Handlungsempfehlungen gegeben werden.
Es wird geprüft ob das Forschungsziel erreicht wurde und wie hoch das Optimierungspotenzial gegenüber den bisherigen Verfahren ist.
Auf dieser Basis wird ein Ausblick auf die Potenziale der betrieblichen Umsetzung gegeben.

Abhängig vom endgültigen Schwerpunkt der Arbeit, könnte die Implementierung der Verfahren ein Unterkapitel "Ausblick" im Rahmen des Ergebnisses werden.
Die Details der Implementierung würden hierdurch aus der Arbeit ausgelagert und abgegrenzt werden.
Sinnvoll könnte dies sein, wenn die primäre Forschungsfrage aufgrund von vielen unterschiedlichen Verfahren einen größeren Anteil der verfügbaren Seitenzahl beansprucht.
Die sekundäre Forschungsfrage wird somit flexibel angepasst, um ausreichend Ressourcen für die primäre Forschungsfrage bereitzustellen.

\subsection{Known Issues}
A) Tagesaktuelle Veränderung der Lage. Aged like Milk.
A2) Probleme bei der wissenschaftlichkeit einiger Passagen.

B) Man bescheinigt Leuten ein PCR-Negativergebnis, für die 3/4 Tests positiv waren.
- Risiko / Fehlerquote
- Verunsicherung
- Nachprüfung und resultierende Kosten

C) Fehler von PCR-Tests / Akzeptable Quote

D) Mögliche Anwendungsfehler

E) XOR / AND bei Algorithmen
Mitigation
A) Veraltung

Tagesaktuelle Lage wird beschrieben und eingeordnet.
Verankerung und Dokumentation im aktuellen Zeitgeschehen
Andere Komponente ist jahrzehnte alt und erprobt. Keine Veränderung.
Methodik kann für weitere Erkrankungen genutzt werden
PCR-Covid19-Tests dürften noch eine Weile erhalten bleiben. Unabhängig der exakten Lage
B) Fehlerquote

Berücksichtigung innerhalb der Konzipierung
Organisatorische Minimierung
\fi

