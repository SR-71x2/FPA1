% !TEX root =  master.tex
\chapter{Ergebnis}
\subsubsection{Erkenntnisse der Arbeit}
Zusammenfassend kann aus der Forschungsarbeit abgeleitet werden:

\begin{itemize}
	\item \textbf{Pooling} Das PCR-Verfahren kann genutzt werden, um mehrere Proben gemeinsam zu testen.
	\item \textbf{Einfache Poolingmethoden} Bereit durch einfache Poolingmethoden mit Nachtestung im Falle eines positiven Pools kann eine deutliche Effizienzsteigerung gegenüber der PCR-Einzeltestung erreicht werden.
	\item \textbf{Komplexe Poolingmethoden} Durch komplexere Analysen und mehrdimensionale Testgruppen kann eine signifikante Effizienzsteigerung gegenüber einfachen Poolingmethode erreicht werden. Insbesondere bei hohen Prävalenzen kann durch komplexe Poolingmethoden ein signifikanter Effizienzgewinn erreicht werden.
	\item \textbf{Potenzial} Das Effizienzsteigerungspotenzial durch Pooling hängt direkt von der Prävalenz der Testgruppe ab. Das Potenzial ist bei geringer Prävalenz besonders hoch.
	\item \textbf{Testort} Vor Ort im Unternehmen sollte nur die Probenentnahme durchgeführt werden. Hintergrund hier sind die effizienteren Methoden und das geschulte Personal, welchen in Laboren zur Verfügung steht.
	\item \textbf{Verzögerung} PCR basierte Verfahren liefern zwangsläufig ein späteres Ergebnis als Schnelltests
	\item \textbf{Erkenntnis} 
	\item CBNAAT könnte perspektivisch PCR ersetzen. Es liefert Ergebnisse schneller und automatisierter. Seine Funktionsweise ist zu PCR ähnlich genug um dieselben Poolingmethoden zu ermöglichen.
	\item Die Nachtestung kann durch mehrstufiges Pooling optimiert werden.
	\item Die Zulassungsstudien der Schnelltests zeigen große Unterschiede in deren Qualität.
	\item Begrenzender Faktor der Testkapazitäten sind die Reaktionen welche im Labor durchgeführt werden.
	\item Für massentestung geringe falsch-positiv Quote essentiell
	\item Übliche Zyklenanzahl XX / Weitere für Pooling
	\item PCR-Ergebnis nach 4-5 Stunden / Sensitivität XX Prozent
	\item Bei Verdachtsüberprüfung über 40 Positiv. Mit den hier betrachteten Methoden kein Pooling sinnvoll.
	\item Verwässerung kein Problem. PCR ist darauf ausgelegt geringe DNA-Menge zu vermehren und nachzuweisen.
	Poolgröße 20 sicher. Bis XX akzeptabel.
	\item Unklare Ergebnisse werden durch Nachtestung der betroffenen Personen überprüft.
	\item Kontamination der Proben muss verhindert werden um Nachtestung zu ermöglichen.
	\item Vergleich Teststrategien mit Erwartungswert für die Effizienz. Effizienz gibt an, wie viele Personen für eine gegebene Menge der knappen Resource "PCR-Kapazitäten" getestet werden können.
	\item Wahrscheinlichkeit eines positivfalles im Pool hängt von Prävalenz und Poolgröße ab. Pools so groß wie möglich wählen, aber klein genug dass die Positivwahrscheinlichkeit im Rahmen bleibt. Bei niedrigen Prävalenzen können große Pools gebildet und die Effizienz start gesteigert werden. Bei höheren Prävalenzen müssen diese Pools aber zu häufig nachgetestet werden.
	

	
\end{itemize}
\cleardoublepage



\subsubsection{Fazit zu den Forschungsfrage}
\textbf{Lässt sich die Effizienz des PCR-Verfahren durch Pooling stark genug steigern, um eine wirtschaftliche Alternative für das Einsatzgebiet von Schnelltests zu bieten?}

Die Hypothese der Arbeit ist, dass PCR durch Pooling effizienter sein kann als Schnelltests.
Hierfür soll überprüft werden, wie sich die Kosten des PCR-Verfahrens durch Pooling entwickeln.
Geprüft werden soll deshalb die Tauglichkeit für eine Massentestung.
Hierbei ist vor allem ein geringer Preis ausschlaggebend.

Diese Hypothese dieser Arbeit wird als bestätigt betrachtet, wenn das PCR-Pooling bei vergleichbaren Kosten wie ein Schnelltest eine höhere Erkennungsrate bietet.

\textbf{Wie könnte eine PCR-basierte Teststrategie im betrieblichen Umfeld realisiert werden?}
Da die primäre Hypothese in Kapitel XX widerlegt wurde, entfällt die Relevanz für die Erarbeitung einer umfassenden betrieblichen Teststrategie.
Trotzdem bleibt festzuhalten, dass das PCR-Verfahren in der Schwerpunkttestung eine erhöhte Zuverlässigkeit gegenüber Schnelltests bietet.
Anstelle einer umfassenden Implementierung werden deshalb Grundsätze für die Durchführung von PCR-Pooling diskutiert.

\if{false}
\subsection{Known Issues}
A) Probleme bei der wissenschaftlichkeit einiger Passagen.

B) Man bescheinigt Leuten ein PCR-Negativergebnis, für die 3/4 Tests positiv waren.
- Risiko / Fehlerquote
- Verunsicherung
- Nachprüfung und resultierende Kosten

C) Fehler von PCR-Tests / Akzeptable Quote
D) Mögliche Anwendungsfehler
\fi

