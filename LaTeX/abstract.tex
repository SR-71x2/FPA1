% !TEX root =  master.tex
\chapter*{Kurzfassung (Abstract)}
\addcontentsline{toc}{chapter}{Kurzfassung (Abstract)}

\subsubsection{Deutsch}
Die vorliegende Arbeit hat zum Ziel, die Effizienz der PCR-Analyse durch Pooling der Proben zu erhöhen.
Beantwortet werden soll die Frage, ob sich die Effizienz des PCR-Verfahren durch Pooling stark genug steigern lässt, um eine wirtschaftliche Alternative für das Einsatzgebiet von Schnelltests zu bieten.
Weiter wird untersucht, wie eine PCR-basierte Teststrategie im betrieblichen Umfeld realisiert werden könnte.

Erläutert werden zunächst Methoden für die Testung und grundlegende Konzepte des Poolings.
Es werden die Effizienzsteigerungspotenziale durch Pooling aufgezeigt und errechnet.

Die Arbeit konnte bereits durch einfach Poolingmethoden ein deutliches Effizienzsteigerungspotenzial bei niedrigen Prävalenzen aufzeigen.
Für hohe Prävalenzen werden komplexere Methoden benötigt, welche nicht Teil dieser Arbeit sind. 
Eine betriebliche Implementierung war nicht relevant, da die Poolingmethoden effizienter im auswertenden Labor angewandt werden können.

\subsubsection{Englisch} %DEEPL.com
The aim of the present work is to increase the efficiency of PCR analysis by pooling samples.
The question to be answered is whether the efficiency of the PCR process can be increased strongly enough by pooling to offer an economical alternative for the field of application of rapid-antigen-tests.
Further, it will be investigated how a PCR-based testing strategy could be implemented in an operational setting.

Methods for testing and basic concepts of pooling are first explained.
The potential for efficiency gains through pooling are shown and calculated.

At low prevalences the work could already show a clear efficiency increase with simple pooling methods.
For high prevalences, more complex methods are required, which are not part of this work. 
An operational implementation was not relevant, since the pooling methods can be applied more efficiently in the evaluating laboratory.