% !TEX root =  master.tex
\chapter{Ergebnis}
\subsubsection{Erkenntnisse der Arbeit}
Zusammenfassend kann aus der Forschungsarbeit abgeleitet werden:

\begin{itemize}
	\item \textbf{Pooling} Das PCR-Verfahren kann genutzt werden, um mehrere Proben gemeinsam zu testen.
	\item \textbf{Einfache Poolingverfahren} Bereit durch einfache Poolingverfahren mit Nachtestung im Falle eines positiven Pools kann eine deutliche Effizienzsteigerung gegenüber der PCR-Einzeltestung erreicht werden.
	\item \textbf{Mehrdimensionale Poolingverfahren} Durch komplexere Analysen und mehrdimensionale Testgruppen kann eine / keine / Im Bereich von ... eine signifikante Effizienzsteigerung gegenüber einfachen Poolingverfahren erreicht werden.
	\item \textbf{Potenzial} Das Effizienzsteigerungspotenzial durch Pooling hängt direkt von der Prävalenz der Testgruppe ab. Das Potenzial ist bei geringer Prävalenz besonders hoch.
	\item \textbf{Testort} Vor Ort im Unternehmen sollte nur die Probenentnahme durchgeführt werden. Hintergrund hier sind die effizienteren Methoden und das geschulte Personal, welchen in Laboren zur Verfügung steht.
	\item \textbf{Verzögerung} PCR basierte Verfahren liefern zwangsläufig ein späteres Ergebnis als Schnelltests
	\item \textbf{Erkenntnis} 
	\item	CBNAAT
	\item Die Nachtestung kann durch mehrstufiges Pooling optimiert werden.
	

	
\end{itemize}
\cleardoublepage

\subsubsection{Fazit zur Forschungsfrage}

\if{false}
\subsection{Known Issues}
A) Probleme bei der wissenschaftlichkeit einiger Passagen.

B) Man bescheinigt Leuten ein PCR-Negativergebnis, für die 3/4 Tests positiv waren.
- Risiko / Fehlerquote
- Verunsicherung
- Nachprüfung und resultierende Kosten

C) Fehler von PCR-Tests / Akzeptable Quote
D) Mögliche Anwendungsfehler
\fi

