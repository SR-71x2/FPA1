% !TEX root =  master.tex
\chapter{Ergebnis}
 \textbf{Erkenntnisse aus der Arbeit:}
\begin{itemize}
	\item \textbf{Anwendungsfall}
	Die Qualität der Schnelltests unterscheidet sich stark nach Hersteller und ist oft mangelhaft.
	Das PCR-Verfahren liefert im Labor nach 4-5 Stunden ein vergleichsweise sehr zuverlässiges Ergebnis.
	Begrenzender Faktor für mehr PCR-Tests sind fehlende Laborkapazitäten.
    Essentiell für jedes häufig genutzte Testverfahren ist, dass die falsch-positiv Quote sehr gering sein muss.
	
	\item \textbf{Pooling}
	Beim PCR-Verfahren können mehrere Proben gemeinsam getestet werden.
	Bereits durch einfache Poolingmethoden kann eine deutliche Effizienzsteigerung gegenüber der PCR-Einzeltestung erreicht werden.
	PCR ist darauf ausgelegt, geringe DNA-Menge zu vermehren und nachzuweisen.
	Verwässerung kein deshalb Problem.
	Eine Poolgröße von 20 Personen ist sicher und liefern über 95 Prozent Erkennungsrate.
	Bei etwas reduzierter Genauigkeit (80 Prozent) sind Pools mit dreistelligen Personenzahlen möglich.
	Bei anlassbezogenen Tests ist die Prävalenz zu hoch für die hier betrachteten Pooling-Methoden.
	Die Effizienz gibt an, wie viele Personen pro Test getestet werden können.
	Sie ermöglicht den Vergleich zwischen Methoden.
	
	\item \textbf{Nachtestung}
	Unklare Ergebnisse werden durch Nachtestung der betroffenen Personen überprüft.
	Hierfür muss die Kontaminierung der Proben  verhindert werden.
	Die Nachtestung von Pools erhöht die Zeitverzögerung des PCR-Verfahrens.
	
	\item \textbf{Testort}
	Vor Ort im Unternehmen sollte nur die Probenentnahme durchgeführt werden.
	Hintergrund hier sind die effizienteren Methoden und das geschulte Personal, welchen in Laboren zur Verfügung steht.
		
	\item \textbf{Potenzial}
	Das Effizienzsteigerungspotenzial durch Pooling hängt direkt von der Prävalenz der Testgruppe ab.
	Das Potenzial ist bei geringer Prävalenz besonders hoch.
	Die Wahrscheinlichkeit einer Infektion in der Testgruppe hängt von der Prävalenz und Poolgröße ab.
	Bei niedrigen Prävalenzen können große Pools gebildet und die Effizienz start gesteigert werden.
	Bei hohen Prävalenzen müssen die Pools kleiner sein, um Nachtestungen zu vermeiden.
	
\cleardoublepage
	
	\item \textbf{Perspektiven} Über die Arbeit hinaus gibt es zahlreiche Optimierungsmöglichkeiten.
	\textbf{CBNAAT} könnte perspektivisch PCR ersetzen. Es liefert Ergebnisse schneller und automatisierter. Seine Funktionsweise ist zu PCR ähnlich genug um dieselben Poolingmethoden zu ermöglichen.
	Durch \textbf{komplexere Poolingmethoden} und mehrdimensionale Testgruppen kann eine signifikante Effizienzsteigerung gegenüber einfachen Poolingmethode erreicht werden.
	Insbesondere bei hohen Prävalenzen kann hierdurch ein signifikanter Effizienzgewinn erreicht werden.
	Die \textbf{Nachtestung} kann durch mehrstufiges Pooling optimiert werden.
\end{itemize}

\subsubsection{Fazit zu den Forschungsfrage}
\textit{\textbf{Lässt sich die Effizienz des PCR-Verfahren durch Pooling stark genug steigern, um eine wirtschaftliche Alternative für das Einsatzgebiet von Schnelltests zu bieten?}}

Entsprechend der Hypothese der Arbeit ist das Pooling von Proben im PCR-Verfahren möglich und es lassen sich signifikante Steigerungspotenziale beobachten.
Wichtig hierfür ist eine niedrige Prävalenz, wie sie bei anlasslosen Testungen und niedrigen Inzidenzen zu erwarten ist.

Bei Inzidenzwerten von 35 oder 50 - wie sie im Sommer 2021 zu beobachten - lässt sich die Effizienz um einen Faktor 22,5 oder 26,9 steigern. Selbst bei einer höheren Inzidenz von 250 lässt sich noch mehr als eine Verzehnfachung erreichen.
Bei hohen Prävalenzen, wie sie in der Verdachtsfallüberprüfung vorkommen, kann mit den beobachteten Methoden kaum Effizienzgewinn erzielt werden. Hier sollten komplexere Poolingverfahren mit überlappenden Pools und mehrstufigen Nachtestungen betrachtet werden.

Ob das Verfahren damit kostengünstiger durchführbar ist als Schnelltests, muss entsprechend der Abgrenzung für den individuellen Anwendungsfall geprüft werden.

\textit{\textbf{Wie könnte eine PCR-basierte Teststrategie im betrieblichen Umfeld realisiert werden?}}

Wie in Kapitel 4.2 ausgeführt, ist es sicherer und wirtschaftlicher das Pooling der Proben nicht im Unternehmen sondern im Labor durchzuführen.
Hierdurch entfällt die Relevanz für eine umfassenden betrieblichen Implementierung.
Die betriebliche Organisation beschränkt sich auf die Entnahme und Beschriftung der Proben, sodass diese ins Labor gesendet und ausgewertet werden können.