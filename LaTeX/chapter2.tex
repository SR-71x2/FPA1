% !TEX root =  master.tex
\chapter{Sekundäre Forschungsfrage und Ausblick}


%#########################################################
%#########################################################
\if{false}
\section{Planung aus RP}
Die sekundäre Forschungsfrage ist die Implementierung der Methode im betrieblichen Umfeld.
Der Umfang dieser Forschung wird flexibel dem Ressourcenbedarf der primären Forschungsfrage angepasst.
Die Behandlung der Implementierung ist somit als eigenes Hauptkapitel denkbar.
Alternativ erfolgt eine Kürzung als Ausblick nach dem Ergebnis.

\section{Obsidian Sammlung}
\subsection{Implementierung}
Bei der Informatik, sinkt der Paritätsbedarf bei einer erhöhten Datenmenge.
Hier ist in der Theorie eine endlose Steigerung der Datenmenge effizient, bis zu einem Punkt an dem mehr Bitfehler wahrscheinlich sind als das System berichtigen kann.

In diesem Kapitel sollen betriebswirtschaftliche Aspekte Beachtung finden, insbesondere Skalierung und Logistik.

Es soll auf Basis der Kenntnisse aus dem Algorithmen-Kapitel mehrere Berechnungsmethoden und Skalierungsfaktoren erarbeitet werden.

Für den Abstrich könnten handelsübliche Schnelltests eingesetzt werden. Damit wird der Abstrich gemacht und der Pufferbehälter genutzt.

Mit 1-2 Tropfen wird ein Schnelltest gemacht. Ist dieser Positiv, wird ein voller PCR-test gemacht. Mit den Restlichen 5-7 Tropfen der Probe kann im Falle eines Negativergebnis in den PCR-Algorithmus eingestiegen werden.

Symptome| Schnelltest postitiv| PCR-Algorithmus
--------|----------|---------
Erste Vorprüfung| |
Wenn Negativ ->|Probenentnahme und Vorcheck|
-- | Wenn negativ -> | Probenversand Labor

Im Selbstversuch waren meist 7-10 Tropfen aus den Schnelltests heraus zu bekommen. 3-4 reichen für den eigentlichen Schnelltest.
\fi