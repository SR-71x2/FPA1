% !TEX root =  master.tex
\chapter{Einleitung}
\section{Problemstellung und Forschungsfragen}
Zum Zeitpunkt dieser Arbeit wird seit über 2 Jahren versucht die Covid19-Pandemie einzudämmen und die Kapazitäten des Gesundheitssystems nicht zu überlasten.
Ein elementarer Baustein ist hierbei die Massentestung der Bevölkerung auf Infektion mit SARS-CoV2.
Hierbei erfolgt die anlasslose Massentestung üblicherweise mit Antigen-Schnelltests, während Verdachtsfälle über PCR-Tests überprüft werden.\footcite{bund_testverordnung_2021}
Unternehmen testen ihre Mitarbeitern, um Infektionen frühzeitig zu erkennen und eine Verbreitung zu vermeiden.
Der Nachweis einer negativen Testung ist zudem - alternativ zu einer Immunisierung - Voraussetzung für die Teilnahmen an vielen Bereichen des öffentlichen Lebens.\footcite{land_corona-verordnung_2022}
Die Genauigkeit der Schnelltests ist allerdings oftmals nicht ausreichend\footcite{wagenhauser_clinical_2021} und weist sehr große Qualitätsunterschiede zwischen den Herstellern auf.\footcite{zerforschung_zerforschung_2022}

Im Frühjahr 2022 wurde die Kapazität für PCR-Testungen durch die stark gestiegenen Infektionszahlen überschritten.
Deshalb wurde die Möglichkeit für PCR-Testungen im Januar 2022 stark eingeschränkt.\footcite{land_corona-verordnung_2022}
Das qualitativ hochwertigere PCR-Verfahren steht seitdem nur noch zur Überprüfung eines positiven Antigen-Schnelltests zur Verfügung.
Diese erkennen allerdings nicht alle Infektionen, sodass viele Personen von einer Erkennung durch PCR ausgeschlossen sind.
Hauptursache hierfür sind begrenzte Kapazitäten in den Laboren.

\textit{"ln most laboratories, the screening capacity is limited by the number of PCR reactions that can be performed in a day. It is, therefore, desirable to maximize the number of samples that can be tested per reaction."}\footcite{viehweger_increased_2020}

Im Laufe der Pandemie wurden von vielen Forschungsgruppen und Laboren Methoden entwickelt, um PCR-Pooling durchzuführen.
Diese machen es möglich, Patienten gemeinsam zu testen und Analysekapazitäten einzusparen.
Ziel der Arbeit ist es, die Kosten einer PCR-Testung zu senken und vorhandene Laborkapazitäten effizienter zu nutzen.

Die Arbeit hat zum Ziel, die folgenden \textbf{Forschungsfragen} zu beantworten:
\begin{itemize}
	\item \textbf{Lässt sich die Effizienz des PCR-Verfahren durch Pooling stark genug steigern, um eine wirtschaftliche Alternative für das Einsatzgebiet von Schnelltests zu bieten?}
	\item \textbf{Wie könnte eine PCR-basierte Teststrategie im betrieblichen Umfeld realisiert werden?}
\end{itemize}

\section{Abgrenzung}
Um dem Umfang dieser Arbeit gerecht zu werden, mussten Einschränkungen bei den Modellen und weitergehende Ansätzen getroffen werden.
Hierdurch wurden vielversprechende Konzepte ausgelassen oder nur kurz erwähnt.
Insbesondere bei hohen Prävalenzen - welche nicht primärer Teil dieser Arbeit sind - sollten diese Konzepte berücksichtigt werden.

Diese Konzepte werden hier aufgelistet und abgegrenzt.
In der Arbeit werden sie erwähnt.

\begin{itemize}
	\item\textbf{Parameter für die Beantwortung der ersten Forschungsfrage}\newline
	Die Hypothese der Arbeit ist, dass PCR durch Pooling kosteneffizienter sein kann als ein Schnelltest.
	Geprüft werden soll deshalb, ob PCR für eine Massentestung geeignet ist.
	Der Fokus liegt hierbei auf einem möglichst geringen Preis.
	
	Die Hypothese könnte als bestätigt betrachtet werden, wenn durch PCR-Pooling vergleichbaren Kosten wie für ein Schnelltest erreichen kann.
	Die detaillierte Kostenstruktur für eine Testmethode ist allerdings von vielen Faktoren abhängig.
	Hierzu zählen neben der betrachteten Laboruntersuchung auch Raum- und Personalkosten für die Probenentnahme, Logistik und Materialkosten.
	
	Die Kosten für die Testung variieren daher stark und sind abhängig vom Anbieter sowie den erbrachten Eigenleistungen im Unternehmen.
	Ein einheitlicher Preis kann aus diesem Grund nicht existieren und eine abschließende Kalkulation ist im Rahmen dieser Arbeit nicht möglich.
	
	Für die Ziele der Arbeit ist es  ausreichend zu prüfen, welcher Effizienzgewinn bei der Laboranalyse durch Pooling erreicht werden kann.
	Die Übertragung dieses Faktors in einen Geldbetrag und Interpretation der Kostenersparnis muss allerdings dem Leser überlassen bleiben und kann nur anhand der lokalen Marktangebote geschehen.
		
	\item\textbf{Abgrenzung des Einsatzgebiets}\newline
	Das gewünschte Einsatzgebiet der anlasslose Massentestung ist ein anderer Anwendungsfall als die Überprüfung von Verdachtsfällen.
	Es ist mit für PCR unüblich niedrigen Prävalenzen zu rechnen.
	Für die Ziele dieser Arbeit ist es somit erforderlich, große Pools zu bilden um die notwendige Kostenreduzierung zu erreichen.
	Aufgrund der Ungenauigkeit der Schnelltests, können einige falsch-negative Ergebnisse akzeptabel sein.
    Symptomatische Personen sollten weiter durch Einzeltests geprüft werden.
	\item\textbf{Keine Berücksichtigung fehlerhafter Testergebnisse und Fehlanwendung}\newline
	Der Umgang mit fehlerhaften Ergebnissen wird in Kapitel 3.2 und Kapitel 4.2 theoretisch diskutiert.
	Mögliche Fehlerquoten fanden allerdings keinen Einzug in die Berechnungsmodelle für die Optimierung.
	Basierend auf den Empfehlungen in Kapitel 4.2 in Verbindung mit den Parametern welche in Kapitel 3.4 für die Optimierung verwendet werden, ist keine unüblich hohe Fehlerquote zu erwarten.
	\item\textbf{Veränderungen der PCR-Zyklen durch Pooling}\newline
	Da gepoolte Proben verwässert sind, müssen für einige Methoden weitere PCR-Zyklen eingeplant werden.
	Die Effizienz aller betrachteten Methoden würde um ungefähr 10 Prozent niedriger ausfallen.
	Dies wurde in den Modellen nicht berücksichtigt.
	\item\textbf{Weiteres Potenzial durch Pooling- und Nachtestungsmethoden}\newline
	Komplexere Poolingmethoden und mehrstufige Nachtestungen haben das Potenzial die Effizienz des Poolings nochmal signifikant zu steigern.
	Insbesondere bei hohe Prävalenzen liefern sie bessere Ergebnisse als die hier betrachteten Methoden.
\end{itemize}
