% !TEX root =  master.tex
\chapter{Einleitung}
\section{Problemstellung}
Die Covid19-Pandemie existiert zum Zeitpunkt dieser Arbeit seit über 2 Jahren.
Mit unterschiedlichsten Maßnahmen wird versucht, die weitere Ausbreitung einzudämmen und die Kapazitäten des Gesundheitssystems nicht zu überlasten.
Neben Impfungen und Masken zählen auch Einschränkungen des öffenlichen Lebens und die Nachverfolgung von Infektionsketten zu den ergriffenen Maßnahmen.\footnote{(Quelle Verordnung)}

Ein weiterer elementarer Baustein der Pandemiestrategie ist die Massentestung der Bevölkerung auf Infektion mit SARS-CoV2.
Hierbei erfolgt die anlasslose Massentestung üblicherweise mit Antigen-Schnelltests, während Verdachtsfälle über PCR-Tests überprüft werden.
\footnote{(Quelle Verordnung)}
Unternehmen testen ihre Mitarbeitern, um Infektionen frühzeitig zu erkennen und eine Verbreitung zu vermeiden.
 Der Nachweis einer negativen Testung ist - alternativ zu einer Immunisierung - Voraussetzung für die Teilnahmen an vielen Bereichen des öffentlichen Lebens.\footnote{(Quelle Verordnung)}

Die Sensitivität der Schnelltests ist allerdings nach aktueller Auffassung nicht ausreichend.
Die Zulassungsstudien der Schnelltests zeigen eine sehr große Bandbreite in der Qualität.
(Hersteller) erkennt selbst bei geringer Viruslast XX Prozent der Infektionen
(Hersteller) dagegen zeigt selbst bei sehr hoher Virenlast nur XX Prozent der Infizierten richtig an.
\footnote{Zerforschung Jan 2022}

Im Frühjahr 2022 wurde die Kapazität für PCR-Testungen durch die Stark gestiegenen Infektionszahlen überschritten.
Deshalb wurde die Möglichkeit für PCR-Testungen im Januar 2022 stark eingeschränkt.\footnote{Quelle neue Testverordnung}
Das qualitativ hochwertigere PCR-Verfahren steht seitdem nur noch zur Überprüfung eines positiven Antigen-Schnelltests zur Verfügung.
Diese erkennen allerdings wie oben ausgeführt nicht alle Infektionen, sodass diese Personen von einer Erkennung durch PCR ausgeschlossen sind.

\section{Zielsetzung und Forschungsfrage}

\textit{"ln most laboratories, the screening capacity is limited by the number of PCR reactions that can be performed in a day. It is, therefore, desirable to maximize the number of samples that can be tested per reaction."}\footnote{Viehweger Z21-23}

Im Laufe der Pandemie wurden von vielen Forschungsgruppen und Laboren Methoden entwickelt, um PCR-Pooling durchzuführen.
Hierdurch wird es möglich, Patienten gemeinsam zu testen und hierdurch Analysekapazitäten einzusparen.
Ziel der Arbeit ist es, die Kosten einer PCR-Testung zu senken und vorhandene Laborkapazitäten effizienter zu nutzen.
Hierfür sollen PCR-Poolingmethoden überprüft und verglichen werden.

%\subsubsection{Forschungsfragen}
Die Arbeit hat zum Ziel, die folgenden \textbf{Forschungsfragen} zu beantworten:

\begin{itemize}
	\item \textbf{Lässt sich die Effizienz des PCR-Verfahren durch Pooling stark genug steigern, um eine wirtschaftliche Alternative für das Einsatzgebiet von Schnelltests zu bieten?}
	\item \textbf{Wie könnte eine PCR-basierte Teststrategie im betrieblichen Umfeld realisiert werden?}
\end{itemize}

Die Hypothese der Arbeit ist, dass PCR durch Pooling effizienter sein kann als Schnelltests.
Hierfür soll überprüft werden, wie sich die Kosten des PCR-Verfahrens durch Pooling entwickeln.
Geprüft werden soll deshalb die Tauglichkeit für eine Massentestung.
Hierbei ist vor allem ein geringer Preis ausschlaggebend.

Um sich hiermit messen zu können, ist es notwendig die Prioritäten der PCR-Methode ähnlich festzulegen.
Für die Ziele dieser Arbeit ist es somit erforderlich, große Pools zu bilden um die notwendige Kostenreduzierung zu erreichen.
Die Präzision fällt hierbei allerdings unter die Schwelle dessen, was üblicherweise für PCR als akzeptabel betrachtet wird.
Durch die bereits diskutierte niedrige Erkennungsrate vieler Schnelltests, kann auch beim PCR-Pooling ein Verlust an Genauigkeit akzeptiert werden.

Diese Hypothese dieser Arbeit wird als bestätigt betrachtet, wenn das PCR-Pooling bei vergleichbaren Kosten wie ein Schnelltest eine höhere Erkennungsrate bietet.

%\subsubsection{Gang der Untersuchung}

%#########################################################
%#########################################################
\if{false}
\section{Planung aus RP}
Die Einleitung beginnt mit einer Beschreibung der Problemstellung und Zielsetzung der Arbeit.
Hierauf folgen die Forschungsfragen, welche sich am Research Proposal orientieren.
Am Ende des Einleitungskapitels wird das PCR-Verfahren im Allgemeinen vorgestellt, um eine Grundlage für die Bearbeitung der Forschungsfragen zu schaffen.

\section{Obsidian Sammlung}
\subsection{Problemstellung:}
Pandemiestatistiken:
- Die Covid19-Pandemie ist zum aktuellen Zeitpunkt im XXX 2022 seit XX Monaten ausgerufen.

Ziel der Maßnahmen
- die weitere Ausbreitung einzudämmen
- die Kapazitäten des Gesundheitssystems in den Wintermonaten nicht zu überlasten.

Maßnahmen:
- Impfungen
- Masken
- Einschränkungen des öffenlichen Lebens
- Nachverfolgung von Infektionsketten

Tests
- anlassbezogene Testung
--- Verdachtsfälle über PCR-Tests
- anlasslose Massentestung der Bevölkerung
--- Antigen-Schnelltests
- Zutrittstests
--- 2G / 3G / 2G+ / Relation Testung
- Tests in Unternehmen
--- Unternehmen sind aktuell verpflichtet ... (aktuelle Verordnung zu Tests)

Die Sensitivität der Schnelltests ist allerdings nach aktueller Auffassung nicht ausreichend. Die Uni Würzburg testete die Sensitivität mehrerer Anbieter in XX/2021 und kam hierbei zu einer Sensitivität von 42,6 Prozent. Fast 6/10 mit Covid19 Infizierte Personen werden durch den Schnelltest somit nicht erkannt.

Aus diesem Grund erfordert die Corona-Verordnung-BW bei einem verstärkten Infektionsgeschehen in vielen Bereichen einen PCR-Test für ungeimpfte Personen. Die Erneute Aufnahme der Testung von geimpften Personen wird diskutiert (Quelle). Allerdings ist die Sensitivität der Schnelltests bei Geimpften Personen aufgrund der niedrigeren Virenlast noch schlechter. (Quelle)

footnote{Die Uni Würzburg testete im XXXX 2021 die Produkte mehrerer Hersteller.
	Das Ergebnis war eine durchschnittliche Sensitivität von 42,6 Prozent.}

Bei geimpften Personen ist die Sensitivität der Schnelltests aufgrund der niedrigeren Virenlast noch schlechter.
footnote{Q: Niedrigere Virenlast Geimpfte}

Während der Erarbeitung dieses Research Proposals wurde beispielsweise eine neue Virusvariante B.1.1.529 "Omikron" in Südafrika entdeckt.
\footnote{Zeitung Omikron}
Ersten Beobachtungen legen nahe, dass Anpassungen an Testverfahren sowie bei den Impfstoffen notwendig werden könnten.
\footnote{Resistenz Omikron}

Bestehende Verfahren sollen analysiert und miteinander vergleichen werden.
Im Laufe der Arbeit soll hierfür ein Kriterienkatalog entwickelt werden.

Im Laufe der Pandemie wurden von vielen Forschungsgruppen und Laboren Methoden entwickelt, um PCR-Pooling durchzuführen. Die Skalierung ist hier sehr unterschiedlich und es gibt widersprüchliche Aussagen dazu, wie robust das PCR Verfahren gegen Verwässerung der Proben ist. Einige behaupten man könne maximal 5 Personen gemeinsam testen. Andere testen 25-40 gemeinsam.

In Deutschland haben die größte Erfahrung die Blutspendedieste zu haben, da diese seit Jahrzehnten Pooling-Verfahren einsetzen um auf HIV und Hepatitis zu testen (Ärtzeblatt). Diese haben hierfür auch ein Patent angemeldet. Die Methode dieses Patents soll die Basis für den Vergleich anderer Verfahren sein.

\subsection{Abgrenzung}
Das Verfahren bricht zusammen, wenn zu viele Personen positiv getestet werden.
Daher eignet es sich nur für anlasslose Tests.
Personen mit Symoptomen müssen auf klassische Weise verifiziert werden. Dies ist auch sicherer.
\fi