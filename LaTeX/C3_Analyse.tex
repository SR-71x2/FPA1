% !TEX root =  master.tex
\chapter{Analyse und Bewertung der Poolingmethoden}
\section{Qualitativ}
\cleardoublepage
\cleardoublepage

\section{Quantitativ}

\cleardoublepage
\cleardoublepage

%#########################################################
%#########################################################
\if{false}
\section{Planung aus RP}
Die Beantwortung der primären Forschungsfrage beginnt mit einer Aufbereitung der bisherigen Forschung zu PCR-Pooling-Verfahren.
Für die Validierung sind zwei Forschungsansätze denkbar:

Qualitativer Ansatz:
Die Disziplin der Kanalcodierung wird vorgestellt und auf Basis dieses wissenschaftlichen Frameworks die Pooling-Verfahren formalisiert.
Hierauf erfolgt eine argumentativ-deduktive Analyse, welche die Methode durch theoretische und qualitative Ansätze überprüft.


\section{Obsidian Sammlung}
\subsubsection{Kanalcodierung}
\begin{itemize}
	\item In der Raumfahrt ist Strahlung eines der Hauptprobleme, welche Bitflips
	\footnote{Die binäre Änderung eines Speicherfeldes}
	in Speichern auslösen kann.
	Hierbei sind oftmals Fehler inakzeptabel, weswegen hochgradig redundante Systeme zum Einsatz kommen.
	Es werden Teilweise ganze Systeme mehrfach verbaut, um die Ergebnisse zu vergleichen.
	\item Ethernetpakete haben dieselbe Herausforderung wie die Raumfahrt, dass durch Störungen Bits verloren gehen können.
	Üblicherwiese sind Ethernetpakete allerdings unkritischer, können erneut gesendet werden und die Strahlungsintensität ist geringer.
	Deshalb werden hier Fehler nur erkannt, aber auf eine Korrektur verzichtet.
	Beschädigte Pakete werden verworfen und müssen erneut gesendet werden.
	\item Anhand der existierenden RAID-Level können die unterschiedlichen Ziele von Kanalcodierung veranschaulicht werden.
	Zwischen Sicherheit, Verfügbarkeit und Berechnungsintensität muss eine Abwägung getroffen werden.
	- Ein System kann wie bei RAID-0 zulasten seiner Integrität beschleunigt werden.
	- Bei RAID-1 wird wie in der Raumfahrt eine volle Redundanz hergestellt. Die hohe Fehlertoleranz der Daten wird hierbei durch hohen Mehraufwand erkauft.
	- RAID-5 und RAID-6 versuchen die Speicherkosten zu optimieren und eine dem Umstand angemessene Datensicherheit zu erreichen. Hierbei wird ein deutlicher Berechnungsaufwand für die Parität, eine lange Rebuild-Zeit und ein gewisses Ausfallrisiko in Kauf genommen.
	\item Der Reed-Solomon-Code, welcher Beispielsweise in CDs eingesetzt wird, ist in der Lage Burst-Errors
	\footnote{Fehler, welche nicht zufällig verteilt sind, sondern in Clustern auftreten.}
	zu erkennen.
	Bei CDs kann dies der Fall sein, wenn diese durch einen zusammenhängenden Kratzer beschädigt ist.
	\item Der Hamming Code war einer der ersten ECC-Algorithmen und wurde in den 1950ern von Richard W. Hamming entwickelt.
	Seine Verteilung der Paritäts-Bits ermöglicht eine effiziente Überprüfung der Daten.
	Durch Verwendung von N+1 Paritäts-Bits, kann die Integrität von 2-hoch-N Daten zu überprüfen werden.
	Die Berichtigung einzelner Bitfehler ist ebenfalls möglich.
	Sollte mehr als ein Fehler innerhalb des Blocks auftreten, kann dieser zwar erkannt, aber nicht berichtigt werden.
	Für einen Speicherbereich mit 256 Bit werden somit 8+1 Paritäts-Bits benötigt, was einem Overhead von nur 3,5 Prozent entspricht.
	Neuere Algorithmen haben die Effizient weiter gesteigert und sollen im Laufe der Arbeit vergleichen werden.
\end{itemize}

\subsubsection{Erstellung eigener Modelle}
Geprüft werden soll die Übertragbarkeit mehrerer in der Informatik gängigen ECC-Algorithmen auf den medizinischen Bereich.
Die Theorie wäre, dass Covid-Infektionen bei anlasslosen Testungen wie auch Bitfehler selten sind.
Somit könnten dieselben Algorithmen zur Effizienzsteigerung genutzt werden.
Ziel ist es, eine Möglichkeit zu finden die exponentielle Effizienzsteigerung von ECC-Algorithmen auf medizinische Testungen anzupassen und hierdurch die Kosten deutlich zu reduzieren.
Hierbei müssen Anpassungen an den Algorithmen vorgenommen werden und neue Herausforderungen beachtet werden.

Beispielsweise sind bei einem 15-11-Hamming-Code nur 11/16 Bits echte Daten.
Die Paritätsbits kosten hier direkt Speicherkapazität.
Bei einer PCR-Testung wären theoretisch alle 16 Plätze verwendbar, da die Durchführung von PCR-Tests selbst (anders als bei Speicher) keine Plätze kostet.
Desweiteren sind neue Probleme zu erwarten, wenn die Tests von Menschen durchgeführt werden.
Hierbei entstehen Fehler, welche in den bisherigen Algorithmen keine Berücksichtigung finden mussten.


\section{Planung aus RP - Quant}
Die Beantwortung der primären Forschungsfrage beginnt mit einer Aufbereitung der bisherigen Forschung zu PCR-Pooling-Verfahren.
Für die Validierung sind zwei Forschungsansätze denkbar:

Quantitativer Ansatz:
Die Pooling-Verfahren werden in Software nachgebaut und quantitativ anhand einer Simulation analysiert.
Es werden unterschiedliche Grenzfälle getestet, um die Auswirkung auf das Verfahren zu beobachten.


\section{Obsidian Sammlung}
\fi