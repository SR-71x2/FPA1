% !TEX root =  master.tex
\chapter{Das PCR Verfahren}
\section{Einordnung des PCR-Verfahrens in der Covid19-Teststrategie}
\subsubsection{Rapid antigen}
Den Schwerpunkt der Teststrategie bietet derzeit der Rapid-Antigen-Schnelltest.
Das Verfahren kann schnell und günstig angewandt werden, was aktuell millionenfach täglich durchgeführt wird.\footnote{Quelle tägliche Kosten Test}
Die Testkits können für unter einen Euro pro Testperson gekauft werden\footnote{Kaufpreis Schnelltests}
und das Ergebnis liegt innerhalb weniger Minuten vor.
Die Probe ist hierbei nach Entnahme nur 60min stabil,\footnote{Quelle Schnelltest nur 60min stabil}
sodass die Auswertung vor Ort erfolgen muss.

Die Qualität der Schnelltests unterscheidet sich deutlich zwischen den Herstellern.
Eine Aufbereitung der Zulassungsstudien ergibt folgende Qualitätsverteilung:\footnote{Zerforschung / Schnelltesttest}

Die Spezifizität der Schnelltests ist mit deutlich über 99 Prozent sehr hoch, was für eine Massentestung auf Bevölkerungsebene essentiell ist, um nicht massenhaft falsch-positive Ergebnisse zu produzieren.\footnote{Quelle Bayessches Theorem}

\subsubsection{PCR}
Das polymerase chain-reaction-Verfahren (PCR) ist seit vielen Jahren der Standard in der Forensik\footnote{Quelle Forensik PCR}
und im Nachweis von Viruserkrankungen. Es bietet eine hohe Erkennungsrate (Sensitivität) und ist für geschultes Personal relativ einfach durchführbar. Notwendig sind allerdings spezielle Geräte, weshalb die Tests üblicherweise in Speziallaboren durchgeführt werden. Die eigentliche Testzeit von 4-5 Stunden wird hierdurch um den Transportweg der Proben verlängert. Das nachfolgende Kapitel beschäftigt sich näher mit diesem Verfahren. 

\subsubsection{TrueNAT and CBNAAT}
60min bis ergebnis
Sens 80-80
Spez 90-95

Cartridge-Based-NAAT nutzt Einweg-Container für die Proben jedes Patienten, welche in ein vollautomatisches Diagnosegerät gefüllt werden können. Hierdurch ist ein sehr hoher Automatisierungsgrad und die Reduzierung von Fehlern erreichbar. Die Methode begann wenige Jahre vor der Pandemie gegen Tuberkulose entwickelt und von den Herstellern zwischenzeitlich auf den neuen Virustyp angepasst.

\footnote{https://factly.in/explainer-what-are-the-different-types-of-tests-being-used-in-india-for-covid-19-detection/  https://www.youtube.com/watch?v=FJFXYDP8N7M}

\subsubsection{IgG Antigen Tests}
Bei einen IgG-Antigen-Test wird der Antikörperspiegel durch eine Blutentnahme ermittelt.
Das Verfahren wird zur Erkennung einer vergangenen Infektion und zur Kontrolle der Impfwirksamkeit eingesetzt.
Zur Diagnostik einer akuten Infektion ist es nicht geeignet, weshalb es für diese Arbeit keine Relevanz hat.
\footnote{Quelle IgG Antikörper / Paper von Lenz-Website}


\cleardoublepage
\section{Funktionsweise des RT-PCR-Verfahrens}
Ct werte und stuff

Durch Flüssigkeit werden Proteine und Fette gelöst / nur RNA bleibt übrig
4-5 Stunden bis ergebnis
90 proben können Zeitgleich getestet werden
Sensitivität: 60-90 Prozent
Spezifizität: 90-95 Prozent

\cleardoublepage

\section{Möglichkeiten und Grenzen zum Pooling bei PCR}
\subsubsection{Unklare Ergebnisse und Nachtestung}
Durch Pooling besteht - abhängig vom Verfahren - das Risiko, dass die Ergebnisse nicht für alle Testpersonen eindeutig interpretiert werden können.
Bei vielen Pooling-Verfahren ergibt sich hierdurch die Notwendigkeit einer Nachtestung.
Ob dies der Fall ist und welche Quote der Testpersonen nachuntersucht werden muss, ist abhängig vom gewählten Verfahren.

Durch die erneute Testung geht Zeit verloren, bevor für alle Testpersonen das Ergebnis fest steht.
Die Proben müssen zudem ausreichend umfangreich sein, um genug Substanz für mehrere Testungen zu enthalten.
Beim Pooling des ersten Durchlaufs muss darauf geachtet werden, die Proben untereinander nicht zu kontaminieren.

\subsubsection{Prävalenz}
Die Prävalenz ist die Quote, mit welcher eine Krankheit in einer Stichprobe vorkommt.
Sie ist ähnlich der derzeit allgemein bekannteren Inzidenz, welche sich auf die Gesamtbevölkerung bezieht.
Bei einer anlasslosen, repräsentativen Testung der Bevölkerung kann die Prävalenz eines Tests gleich der Inzidenz sein.

Bei einer anlassbezogenen Testung werden allerdings meist deutlich höhere Prävalenzen beobachtet.

\footnote{Leon Gordis S37}
\footnote{Beispiel Zeitungsbericht 70 Prozent Prävalenz}


\subsubsection{Verhinderung von Kontamination}
Die komplette Matrix sollte vor dem Pooling einmal dupliziert werden. 
Die für den aktuellen Test notwendigen Proben werden hierbei entnommen und im Duplikat gepoolt.
Für diesen Duplikationsschritt gibt es spezialisierte Laborgeräte, sodass dies in einem Arbeitsschritt für alle Proben durchgeführt werden kann.

%Doppelt
Hierfür muss zu beginn genug Probenmaterial bereit stehen und dieses darf nicht bei der Kombination kontaminiert werden.
Es empfiehlt sich, die Testmatrix zu beginn einmal zu klonen, um in der Originalmatrix ohne kontamination einzeln nachtesten zu können.

\subsubsection{Mögliche Poolgrößen und Erkennungsrate}
Um ein zuverlässiges Ergebnis zu liefern, dürfen die Proben nicht zu stark verwässert werden.
Hierbei wird empfohlen, maximal 20 Personen in einem Pool zu kombinieren. \footnote{Vieweger v1}
Die Testgruppe kann je nach Verfahren größer sein, solange kein Pool mehr als 20 Personen enthält.
Diese Poolgröße liegt laut Viehweger "comfortable above the detection rate"\footnote{Vieweger v1}

Eine höhere Verdünnung ist zulasten der Erkennungsrate problemlos möglich.
Abgewogen werden muss hierbei die Priorisierung zwischen Präzision und Kostenersparnis.



