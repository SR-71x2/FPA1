% !TEX root =  master.tex
\chapter{Implementierung in der betrieblichen Teststrategie}
\section{Ort der Testdurchführung}

\subsubsection{Pooling im Unternehmen}
Das Pooling wird bei diesem Ansatz von Mitarbeitern des Unternehmens durchgeführt.
Das Labor muss nicht einmal zwangsläufig wissen, dass Pooling durchgeführt wird.
\footnote{Abhängig vom Grad der Verwässerung sollte diese Information mitgeteilt werden, um die Anzahl der Zyklen zu erhöhen.}

Keine Verarbeitung der Proben im Unternehmen aufgrund 
\begin{itemize}
	\item Unsachgemäße Handhabung
	\subitem Risiko der Ansteckung
	\subitem Risiko von fehlerhafter Verarbeitung
	\subitem Risiko der Kontamination der Probe
	\item Effizenz
	\subitem Geeigente Geräte im Labor
	\subitem Höhere Geschwindigkeit
	\subitem Routine in der Anwendung
\end{itemize}

Von einer Probenverarbeitung im Unternehmen wird deshalb abgeraten.

\subsubsection{Pooling im Labor}
Beim Pooling im Labor werden im Unternehmen nur die Proben entnommen, beschriftet und an das Labor gesendet.
Hierdurch wird Arbeitsaufwand an das Labor verlagert und es wird ein Labor benötigt, welches das Pooling anbietet.

Der deutliche Vorteil ist hierbei, dass das Pooling von Medizinisch geschultem Personal mit angemessenen Werkzeugen durchgeführt wird.
Hierdurch ist von einer geringeren Fehlerquote, höherer Effizienz und einer Risikoreduktion im Umgang mit den möglicherweise kontaminierten Proben auszugehen.
Eine Durchführung des Poolings im Labor wird deshalb nach Möglichkeit empfohlen.

% Hier entfernen ans Ende der Pooling-Entwicklung?
Auf der Evaluierung der Poolingmethoden wurden mathematisch sinnvolle Verfahren für die jeweiligen Inzidenzstufen ermittelt.
Diese sollen nun um weitere Parameter erweitert werden, um ihre Tauglichkeit im betrieblichen Umfeld zu ermitteln.


\section{Skalierung im betrieblichen Umfeld}
Skalierung im Betrieblichen Umfeld funktioniert grundsätzlich anders als in der Informatik.
Während große Speicherblöcke den Paritätsbedarf senken, ergibt sich durch die Vergrößerung einer Testgruppe ein deutlicher Mehraufwand an Logistik und Organisation. 
Diese Aspekte sollen im vorliegenden Kapitel Beachtung finden.

Grundsätzlich gibt es zwei Ansätze, das Pooling zu organisieren, welche nachfolgend kurz beschrieben werden.


\subsubsection{Organisation im Unternehmen}
Mitarbeiter bekommen persistenten Voucher-Barcode auf dem Alle Daten und auch Abteilung / Kontaktpersonen hinterlegt sind
Ggf. Kontaktpersonen über Plattform oder auf Zettel mit Nr selbst angeben.

Tests werden an MA verteilt oder zentral im Gebäude entnommen.
Die Teströhren bekommen einen Barcode und gehen unverändert ins Labor.

Die Probenentnahme muss von einer geschulten Person beaufsichtigt werden, um Fehlanwendung und Missbrauch zu verhindert.
Hierfür gibt es in vielen Betrieben bereits Personal, welches nach \footnote{§XXXX}
für die Beaufsichtigung der 3G-Nachweise zugelassen ist.

Die Teströhrchen müssen bereits im Unternehmen beschriftet werden, um die Ergebnisse später zuzuordnen.
Hierbei bietet es sich an, eine nicht datenschutzrelevante Liste mit Personalnummern zu verwenden.
Ggf. kann in Büros die Telefondurchwahl als Testnummer genutzt werden.


Müll durch Einmaltests beachten. ggf Glasröhrchen für Proben und abkochen.

\subsubsection{Politische Ebene}
Politisch ist anzumerken, dass bei den Corona-Schutzmaßnahmen zwischen präzissen PCR-Tests und ungenauen Schnelltests getrennt wird.
Die Teilnahme an einigen Veranstaltungen ist somit nur mit PCR-Test zulässig.
es kann unterstellt werden, dass der Gesetzgeber hierbei kein oder nur ein schwaches Pooling eingerechnet hat.
Sollte durch die gewählte Pooling-Methode die Genauigkeit deutliche reduziert sein, sollten deshalb nur Schnelltest-Bescheinigungen an die negativ getesteten Personen ausgestellt werden.
\cleardoublepage




%#########################################################
%#########################################################
\if{false}
\section{Planung aus RP}
Die sekundäre Forschungsfrage ist die Implementierung der Methode im betrieblichen Umfeld.
Der Umfang dieser Forschung wird flexibel dem Ressourcenbedarf der primären Forschungsfrage angepasst.
Die Behandlung der Implementierung ist somit als eigenes Hauptkapitel denkbar.
Alternativ erfolgt eine Kürzung als Ausblick nach dem Ergebnis.

\section{Obsidian Sammlung}
\subsection{Implementierung}
Bei der Informatik, sinkt der Paritätsbedarf bei einer erhöhten Datenmenge.
Hier ist in der Theorie eine endlose Steigerung der Datenmenge effizient, bis zu einem Punkt an dem mehr Bitfehler wahrscheinlich sind als das System berichtigen kann.

In diesem Kapitel sollen betriebswirtschaftliche Aspekte Beachtung finden, insbesondere Skalierung und Logistik.

Es soll auf Basis der Kenntnisse aus dem Algorithmen-Kapitel mehrere Berechnungsmethoden und Skalierungsfaktoren erarbeitet werden.

Für den Abstrich könnten handelsübliche Schnelltests eingesetzt werden. Damit wird der Abstrich gemacht und der Pufferbehälter genutzt.

Mit 1-2 Tropfen wird ein Schnelltest gemacht. Ist dieser Positiv, wird ein voller PCR-test gemacht. Mit den Restlichen 5-7 Tropfen der Probe kann im Falle eines Negativergebnis in den PCR-Algorithmus eingestiegen werden.

Symptome| Schnelltest postitiv| PCR-Algorithmus
--------|----------|---------
Erste Vorprüfung| |
Wenn Negativ ->|Probenentnahme und Vorcheck|
-- | Wenn negativ -> | Probenversand Labor

Im Selbstversuch waren meist 7-10 Tropfen aus den Schnelltests heraus zu bekommen. 3-4 reichen für den eigentlichen Schnelltest.
\fi