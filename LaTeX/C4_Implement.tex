% !TEX root =  master.tex
\section{Betriebliche Implementierung}
\textbf{Pooling im Unternehmen}\newline
Das Pooling wird bei diesem Ansatz von Mitarbeitern des Unternehmens durchgeführt.
Das Labor muss nicht einmal zwangsläufig wissen, dass Pooling durchgeführt wird.
Abhängig vom Grad der Verwässerung sollte diese Information allerdings mitgeteilt werden, um die Anzahl der Zyklen zu erhöhen.\footcite{wink_pcr_1994}

Von einer Probenverarbeitung im Unternehmen birgt mehrere Risiken, welche durch eine Auswertung im Labor vermieden werden können.

\begin{itemize}
	\item \textbf{Unsachgemäße Handhabung}\newline
	Für Mitarbeiter besteht ein Ansteckungsrisiko und die Proben könnten durch fehlerhafte Verarbeitung kontaminiert oder zerstört werden.
	\item \textbf{Effiziente Verarbeitung}\newline
	Im Labor stehen geeignete Geräte und erfahrenes Personal zur Verfügung.\footcite{clewley_polymerase_1995}
	Durch die Routine können eine höhere Geschwindigkeit und Qualität erreicht werden.
\end{itemize}

Eine Verarbeitung im Unternehmen wird deshalb grundsätzlich nicht empfohlen.
Unternehmen mit der Möglichkeit ein vollwertiges Labor einzurichten, sind hiervon ausgenommen.

\textbf{Pooling im Labor}\newline
Beim Pooling im Labor werden im Unternehmen nur die Proben entnommen, beschriftet und an das Labor gesendet.
Hierdurch wird Arbeitsaufwand an das Labor verlagert und es wird ein Labor benötigt, welches das Pooling anbietet.
Das Pooling wird hierdurch von medizinisch geschultem Personal mit angemessenen Werkzeugen durchgeführt.
Eine Durchführung des Poolings im Labor wird deshalb empfohlen.

\cleardoublepage

\textbf{Organisation im Unternehmen}\newline
Bei der Auswahl der Poolgröße muss auf die Praktikabilität von Logistik und Organisiation geachtet werden.

Bei Einführung des Systems könnte man allen Mitarbeitern Klebetiketten mit personalisiertem Barcode zusenden.
Wenn die Person an einer Testung teilnimmt, bringt sie den Codeaufkleber mit und dieser wird auf das Teströhrchen geklebt.
Die Tests werden gesammelt an das Labor gesendet und dort ausgewertet.
Zurück übermittelt werden die Testergebnisse nach Barcodenummer.
Die Ergebnisse können vom Unternehmen über die Nummer einer Person zugeordnet werden.
Personenbezogene Daten verlassen nach diesem System nicht das Unternehmen.
Zur Ermittlung von Kontaktpersonen könnten Abteilungen herangezogen werden oder die Mitarbeiter pflegen die Kontaktlisten selbst.

Die Probenentnahme muss von einer geschulten Person beaufsichtigt werden, um Fehlanwendung und Missbrauch zu verhindert.
Hierfür gibt es in vielen Betrieben bereits Personal, welches für die Beaufsichtigung der 3G-Nachweise zugelassen ist.\footcite{bund_testverordnung_2021}
Die Abfallmenge ließe sich durch reinigungsfähige Glasröhrchen und einen Flüssigkeitsbehälter gering halten.

\textbf{Ausstellung von Zerftikaten}\newline
In den Corona-Verordnungen wird unterschieden zwischen Schnelltests und PCR-Tests.
Mit dem Nachweis eines negativen PCR-Tests ist es möglich, zutritt zu Veranstaltungen und Geschäften zu erhalten.
Der Gesetzgeber trennt hierbei klar zwischen den präzisen PCR-Tests und ungenauen Schnelltests.\footcite{bund_testverordnung_2021}
Es ist anzunehmen, dass der Gesetzgeber hierbei kein oder nur ein schwaches Pooling eingerechnet hat.

Durch das Pooling großer Gruppen, könnte die Erkennungsgenauigkeit des PCR-Tests reduziert sein.\footcite{verwilt_evaluation_2021}
In diesen Fällen sollten nur Schnelltest-Bescheinigungen an die negativ getesteten Personen ausgestellt werden.

\cleardoublepage