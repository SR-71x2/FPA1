% !TEX root =  master.tex
\chapter{Das PCR Verfahren}
\section{Übersicht der Covid19-Teststrategie}
\textbf{Rapid-Antigen-Schnelltest} eignen sich durch geringe Kosten und schnelle Ergebnisse für eine Massentestung auf Bevölkerungsebene.\footnote{Quelle tägliche Kosten Test}
Ihre Qualität unterscheidet sich allerdings deutlich zwischen den Herstellern.\footnote{Zerforschung / Schnelltesttest}
Die Quote von falsch-positiven Ergebnissen ist sehr niedrig, was für eine Massentestung auf Bevölkerungsebene essentiell ist.\footnote{Quelle Bayessches Theorem}
Die Probe ist hierbei nach Entnahme nur 60min stabil,\footnote{Quelle Schnelltest nur 60min stabil}
sodass die Auswertung vor Ort erfolgen muss.
Ein positiver Schnelltest ist Voraussetzung für die Teilnahme am PCR-Verfahren.\footnote{Quelle Verordnung}

Das \textbf{PCR-Verfahren} ist seit vielen Jahren der Standard in der Forensik\footnote{Quelle Forensik PCR}
und im Nachweis von Viruserkrankungen. Es bietet eine hohe Erkennungsrate und ist für geschultes Personal relativ einfach durchführbar. Notwendig sind allerdings spezielle Geräte, weshalb die Tests üblicherweise nicht vor Ort sondern in Laboren durchgeführt werden. Die eigentliche Testzeit von 4-5 Stunden wird hierdurch um den Transportweg der Proben verlängert.

\textbf{Cartridge-Based-NAAT} ist ein Verfahren, welches mit zu PCR vergleichbarer Präzision bereits 60 Minuten ein Ergebnis liefert. Es nutzt Einweg-Container für die Proben jedes Patienten, welche durch ein vollautomatisches Diagnosegerät verarbeitet werden. Der hohe Automatisierungsgrad soll die Reduzierung von Kosten und Fehlern ermöglichen. Die Methode wurde wenige Jahre vor der Pandemie gegen Tuberkulose entwickelt und zwischenzeitlich auf den neuen Virustyp angepasst.
\footnote{60min bis ergebnis, Sens 80-80, Spez 90-95 Quelle Auskommentiert} %https://factly.in/explainer-what-are-the-different-types-of-tests-being-used-in-india-for-covid-19-detection/  %https://www.youtube.com/watch?v=FJFXYDP8N7M

\textbf{IgG Antigen Tests} messen durch eine Blutentnahme den Spiegel der neutralisierenden Antikörper.
Das Verfahren wird zur Erkennung einer vergangenen Infektion und zur Kontrolle der Impfwirksamkeit eingesetzt.
Zur Diagnostik einer akuten Infektion ist es nicht geeignet, weshalb es für diese Arbeit keine Relevanz hat.
\footnote{Quelle IgG Antikörper / Paper von Lenz-Website}

\cleardoublepage

\section{Funktionsweise des PCR-Verfahrens}
Das PCR-Verfahren (ploymerase-chain-reaction) hat zum Ziel, das Vorhandensein einer Gensequenz in einer Probe nachzuweisen.
Im Zusammenhang mit der Pandemie wird versucht, die Gensequenz von SARS-CoV2 in der Speichelprobe eines Patienten nachzuweisen, was auf eine akute Infektion hindeuten würde.\footnote{Quelle Interpretation}

Die Probe wird hierfür nach der Entnahme zunächst in einer Flüssigkeit gelöst, welche Proteine und Fette auflöst.\footnote{Quelle Trägerflüssigkeit}
Hierdurch wird in einer positiven Probe die Hülle des Virus aufgelöst, sodass dessen RNA frei in der Flüssigkeit treibt.
Die Flüssigkeit wird dann in mehreren Zyklen erhitzt und abgekühlt.
Durch das Erhitzen verliert der DNA-Doppelstrang seine Wasserstoffbrückenbindung und löst sich zu zwei RNA-Einzelsträngen.\footnote{Quelle Auswirkung erhitzen}
Diese liegen anschließend einzeln vor und können nach dem Abkühlen von der RNA-Polymerase wieder zu einem Doppelstrang vervollständigt werden.\footnote{Quelle Polymerase}
Die DNA-Menge wird hierdurch in jedem Zyklus verdoppelt.
Ziel dieser Polymerase-Kettenreaktion ist es, die Ziel-DNA durch genug Zyklen so lange zu vermehren, bis eine messbare Menge vorliegt.
Die Anzahl der Zyklen wird dabei als ct-Wert angegeben.
Eine gängige Zyklenanzahl sind XX Verdopplungsschritte, wobei nicht immer eine exakte Verdopplung stattfindet.\footnote{Quelle ct Wert}

Das gesamte Verfahren benötigt im Falle von SARS-CoV2 üblicherweise vier bis fünf Stunden bis genug Genmaterial für den Nachweis vorliegt.\footnote{Quelle Dauer}
Die Erkennungsrate (Sensitivität) des Verfahrens liegt bei XX Prozent.\footnote{Quelle Sensitivität PCR}

\cleardoublepage

