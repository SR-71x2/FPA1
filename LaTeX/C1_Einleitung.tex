% !TEX root =  master.tex
\chapter{Einleitung}
\section{Problemstellung}
Die Covid19-Pandemie existiert zum Zeitpunkt dieser Arbeit seit über 2 Jahren.
Mit unterschiedlichsten Maßnahmen wird versucht, die weitere Ausbreitung einzudämmen und die Kapazitäten des Gesundheitssystems nicht zu überlasten.
Neben Impfungen und Masken zählen auch Einschränkungen des öffenlichen Lebens und die Nachverfolgung von Infektionsketten zu den ergriffenen Maßnahmen.\footnote{(Quelle Verordnung)}

Ein elementarer Baustein der Pandemiestrategie ist zudem die Massentestung der Bevölkerung auf Infektion mit SARS-CoV2.
Hierbei erfolgt die anlasslose Massentestung üblicherweise mit Antigen-Schnelltests, während Verdachtsfälle über PCR-Tests überprüft werden.\footnote{(Quelle Verordnung)}
Unternehmen testen ihre Mitarbeitern, um Infektionen frühzeitig zu erkennen und eine Verbreitung zu vermeiden.
 Der Nachweis einer negativen Testung ist - alternativ zu einer Immunisierung - Voraussetzung für die Teilnahmen an vielen Bereichen des öffentlichen Lebens.\footnote{(Quelle Verordnung)}

Die Genauigkeit der Schnelltests ist allerdings oftmals nicht ausreichend.
Die Zulassungsstudien der Schnelltests zeigen sehr große Unterschiede in der Qualität.
(Hersteller) erkennt selbst bei geringer Viruslast XX Prozent der Infektionen
(Hersteller) dagegen zeigt selbst bei sehr hoher Virenlast nur XX Prozent der Infizierten richtig an.
\footnote{Zerforschung Jan 2022}

Im Frühjahr 2022 wurde die Kapazität für PCR-Testungen durch die stark gestiegenen Infektionszahlen überschritten.
Deshalb wurde die Möglichkeit für PCR-Testungen im Januar 2022 stark eingeschränkt.\footnote{Quelle neue Testverordnung}
Das qualitativ hochwertigere PCR-Verfahren steht seitdem nur noch zur Überprüfung eines positiven Antigen-Schnelltests zur Verfügung.
Diese erkennen allerdings wie oben ausgeführt nicht alle Infektionen, sodass viele Personen von einer Erkennung durch PCR ausgeschlossen sind.

\section{Zielsetzung und Forschungsfrage}

\textit{"ln most laboratories, the screening capacity is limited by the number of PCR reactions that can be performed in a day. It is, therefore, desirable to maximize the number of samples that can be tested per reaction."}\footnote{Viehweger Z21-23}

Im Laufe der Pandemie wurden von vielen Forschungsgruppen und Laboren Methoden entwickelt, um PCR-Pooling durchzuführen.
Diese machen es möglich, Patienten gemeinsam zu testen und Analysekapazitäten einzusparen.
Ziel der Arbeit ist es, die Kosten einer PCR-Testung zu senken und vorhandene Laborkapazitäten effizienter zu nutzen.
Hierfür soll das PCR-Poolingmethoden betrachtet werden.

%\subsubsection{Forschungsfragen}
Die Arbeit hat zum Ziel, die folgenden \textbf{Forschungsfragen} zu beantworten:

\begin{itemize}
	\item \textbf{Lässt sich die Effizienz des PCR-Verfahren durch Pooling stark genug steigern, um eine wirtschaftliche Alternative für das Einsatzgebiet von Schnelltests zu bieten?}
	\item \textbf{Wie könnte eine PCR-basierte Teststrategie im betrieblichen Umfeld realisiert werden?}
\end{itemize}

Die Hypothese der Arbeit ist, dass PCR durch Pooling effizienter sein kann als Schnelltests.
Hierfür soll überprüft werden, wie sich die Kosten des PCR-Verfahrens durch Pooling entwickeln.
Geprüft werden soll deshalb die Tauglichkeit für eine Massentestung.
Hierbei ist vor allem ein geringer Preis ausschlaggebend.

Diese Hypothese dieser Arbeit wird als bestätigt betrachtet, wenn das PCR-Pooling bei vergleichbaren Kosten wie ein Schnelltest eine höhere Erkennungsrate bietet.

\cleardoublepage
\section{Abgrenzung}
Um dem Umfang dieser Arbeit gerecht zu werden, mussten einige Einschränkungen an den Modellen getroffen und weitergehende Ansätze ausgelassen werden.
Diese werden hier kurz aufgeführt und an geeigneter Stelle in der Arbeit als Ausblick erwähnt.

\begin{itemize}
	\item\textbf{Einsatzgebiet}\newline
	Das gewünschte Einsatzgebiet ist der Ersatz von Schnelltests anlasslose Massentestung geeignet ist.
	Die Überprüfung von Verdachtsfällen ist als Anwendungsfall zu trennen.
	Hierdurch ist mit für PCR unüblich niedrigen Prävalenzen zu rechnen.
	Um sich mit Schnelltests messen zu können, ist es notwendig die Prioritäten der PCR-Methode ähnlich festzulegen.
	Für die Ziele dieser Arbeit ist es somit erforderlich, große Pools zu bilden um die notwendige Kostenreduzierung zu erreichen.
	Durch die bereits diskutierte niedrige Erkennungsrate vieler Schnelltests, kann auch beim PCR-Pooling ein Verlust an Genauigkeit akzeptiert werden.
	Die Kostenoptimierung steht hierbei im Vordergrund, sodass einige falsch-negative Ergebnisse akzeptabel sein können.
	Aus Gründen der Sicherheit und Geschwindigkeit sollten Personen mit Symoptomen durch klassische Einzeltests verifiziert werden.
	\item\textbf{Fehlerhafte Testergebnisse und Fehlanwendung}\newline
	Der Umgang mit fehlerhaften Ergebnissen wird in Kapitel 3.2 und Kapitel 4.2 theoretisch diskutiert.
	Mögliche Fehlerquoten fanden allerdings keinen Einzug in die Berechnungsmodelle für die Optimierung.
	Basierend auf den Empfehlungen in Kapitel 4.2 in Verbindung mit den Parametern welche in Kapitel 3.4 für die Optimierung verwendet werden, ist keine unüblich hohe Fehlerquote zu erwarten.
	\item\textbf{Veränderungen der PCR-Zyklen durch Pooling}\newline
	Da gepoolte Proben verwässert sind, müssen für einige Methoden weitere PCR-Zyklen eingeplant werden.
	Die Effizienz aller betrachteten Methoden würde um ungefähr 10 Prozent niedriger ausfallen.
	Dies wurde in den Modellen nicht berücksichtigt.
	\item\textbf{Potenzial durch mehrstufige Testung}\newline
	Bei Testumgebungen in denen hohe Prävalenzen zu erwarten sind, kann eine mehrstufige Testung sinnvoller als eine Einzelnachtestung sein.
	\item\textbf{Potenzial durch komplexere Poolingmethoden}\newline
	Komplexere Poolingmethoden haben das Potenzial die Effizienz des Poolings nochmal signifikant zu steigern.
\end{itemize}
