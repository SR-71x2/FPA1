% !TEX root =  master.tex
\chapter{Primäre Forschungsfrage - Quantitativ}



%#########################################################
%#########################################################
\if{false}
\section{Planung aus RP}
Die Beantwortung der primären Forschungsfrage beginnt mit einer Aufbereitung der bisherigen Forschung zu PCR-Pooling-Verfahren.
Für die Validierung sind zwei Forschungsansätze denkbar:
\begin{itemize}
	\setlength{\itemsep}{-8pt}
	\item Qualitativer Ansatz:
	Die Disziplin der Kanalcodierung wird vorgestellt und auf Basis dieses wissenschaftlichen Frameworks die Pooling-Verfahren formalisiert.
	Hierauf erfolgt eine argumentativ-deduktive Analyse, welche die Methode durch theoretische und qualitative Ansätze überprüft.
	\item Quantitativer Ansatz:
	Die Pooling-Verfahren werden in Software nachgebaut und quantitativ anhand einer Simulation analysiert.
	Es werden unterschiedliche Grenzfälle getestet, um die Auswirkung auf das Verfahren zu beobachten.
\end{itemize}

\section{Obsidian Sammlung}
\fi