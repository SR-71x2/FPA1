%%%%%%%%%%%%%%%%%%%%%%%%%%%%%%%%%%%%%%%%%%%%%%%%%%%%%%%%%%
%   Autoren:
%   Prof. Dr. Bernhard Drabant
%   Prof. Dr. Dennis Pfisterer
%   Prof. Dr. Julian Reichwald
%%%%%%%%%%%%%%%%%%%%%%%%%%%%%%%%%%%%%%%%%%%%%%%%%%%%%%%%%%

%%%%%%%%%%%%%%%%%%%%%%%%%%%%%%%%%%%%%%%%%%%%%%%%%%%%%%%%%%
%	ANLEITUNG: 
%   1. Ersetzen Sie firmenlogo.jpg im Verzeichnis img
%   2. Passen Sie alle Stellen im Dokument an, die mit 
%      @stud 
%      markiert sind
%%%%%%%%%%%%%%%%%%%%%%%%%%%%%%%%%%%%%%%%%%%%%%%%%%%%%%%%%%

%%%%%%%%%%%%%%%%%%%%%%%%%%%%%%%%%%%%%%%%%%%%%%%%%%%%%%%%%%
%	ACHTUNG: 
%   Für das Erstellen des Literaturverzeichnisses wird das 
%   modernere Paket biblatex in Kombination mit biber 
%   verwendet - nicht mehr das ältere Paket BibTex!
%
%   Bitte stellen Sie Ihre TeX-Umgebung entsprechend ein (z.B. TeXStudio): 
%   Einstellungen --> Erzeugen --> Standard Bibliographieprogramm: biber
%%%%%%%%%%%%%%%%%%%%%%%%%%%%%%%%%%%%%%%%%%%%%%%%%%%%%%%%%%

\documentclass[fontsize=12pt,BCOR=3mm,DIV=12,parskip=half,listof=totoc,
               paper=a4,toc=bibliography,pointlessnumbers]{scrreprt}
               
               %toc=listof,listof=entryprefix,
               
\makeindex

%% Elementare Pakete, Konfigurationen und Definitionen werden geladen (gegebenenfalls anpassen)
\input{config}

%%
%% @stud
%%
%% PERSÖNLICHE ANGABEN (BITTE VOLLSTÄNDIG EINGEBEN zwischen den Klammern: {...})
%%

%% Arbeit Individuell
\ArtDerArbeit{Forschungsprojektarbeit 1} % "Bachelor" oder "Projekt" wählen
\TitelDerArbeit{Optimierung von Testkapazitäten durch PCR-Pooling}
\WissBetreuer{Prof. Dr. Martin}
\Bearbeitungszeitraum{14.12.2021 -- 14.02.2022}
\Abgabedatum{14.12.2022}

%% Hochschule
\Kurs{Wirtschaftsinformatik}
\Studiengangsleiter{Prof. Dr. Martin, Prof. Dr. Kessel}

%% Student
\AutorDerArbeit{Daniel Jacobi}
\Matrikelnummer{8041730}

%% Firma
\Firma{Volksbank Backnang eG}
\Abteilung{Marktfolge Aktiv - Firmenkunden}
\FirmenBetreuer{Herr Stephan Denz}

%%
%% @stud
%%
%% BIBLIOGRAPHY (@stud: Bibliographie-Stil wählen - Position und Indizierung)
%%  Auswahl zwischen: 
%%   NUMERIC Style
%%   IEEE Style
%%   ALPHABETIC Style
%%   HARVARD Style
%%   CHICAGO Style 
%%   (oder eigenen zulässigen Stil wählen) 
%%
%%%%%%%%%%%%%
%% Zitierstil
%%%%%%%%%%%%%
% NUMERIC Style - e. g. [12]
%\newcommand{\indextype}{numeric} 
%
% IEEE Style - numeric kind of style 
%\newcommand{\indextype}{ieee} 
%
% ALPHABETIC Style - e. g. [AB12]
%\newcommand{\indextype}{alphabetic} 
%
% HARVARD Style 
%\newcommand{\indextype}{apa} 
%
% CHICAGO Style 
%\newcommand{\indextype}{authoryear}
%
%%%%%%%%%%%%%%%%%%%%%%
%% Position des Zitats
%%%%%%%%%%%%%%%%%%%%%%
%\newcommand{\position}{inline} 
%
% (!!) FOOTNOTE POSITION NOT RECOMMENDED IN MINT DOMAIN:
%\newcommand{\position}{footnote}

%% Final: Setzen des Zitierstils und der Zitatposition
\usepackage[
backend=biber,
autocite=footnote,
bibstyle=apa, %Kopiert von RP
style=apa
]{biblatex} 	
\settingBibFootnoteCite
\addbibresource{bibliography.bib} %Kopiert von RP

\usepackage{wrapfig}
\usepackage{subfig}

%%
%% Definitionen und Commands
%%
\newcommand{\abs}{\par\vskip 0.2cm\goodbreak\noindent}
\newcommand{\nl}{\par\noindent}
\newcommand{\mcl}[1]{\mathcal{#1}}
\newcommand{\nowrite}[1]{}
\newcommand{\NN}{{\mathbb N}}
\newcommand{\imagedir}{img}

\makeindex

\begin{document}

\setTitlepage

%%%%%%%%%%%%%%%%%%%%%%%%%%%%%%%%%%%
% EHRENWÖRTLICHE ERKLÄRUNG
%
% @stud: ewerkl.tex bearbeiten
%
%\input{ewerkl} 
%\cleardoublepage  
%%%%%%%%%%%%%%%%%%%%%%%%%%%%%%%%%%%

%%%%%%%%%%%%%%%%%%%%%%%%%%%%%%%%%%%
% SPERRVERMERK
%
% @stud: nondisclosurenotice.tex bearbeiten
%
%\input{nondisclosurenotice} 
%\cleardoublepage
%%%%%%%%%%%%%%%%%%%%%%%%%%%%%%%%%%%

%%%%%%%%%%%%%%%%%%%%%%%%%%%%%%%%%%%
%	KURZFASSUNG
%
% @stud: acknowledge.tex bearbeiten
%
%\input{acknowledge}
%\cleardoublepage 
%%%%%%%%%%%%%%%%%%%%%%%%%%%%%%%%%%%

%%%%%%%%%%%%%%%%%%%%%%%%%%%%%%%%%%%
% VERZEICHNISSE und ABSTRACT
%
% @stud: ggf. nicht benötigte Verzeichnisse auskommentieren/löschen
%
\tableofcontents
\cleardoublepage

% Abbildungsverzeichnis
\phantomsection
\addcontentsline{toc}{chapter}{\listfigurename}
\listoffigures
\cleardoublepage

%	Tabellenverzeichnis
%\phantomsection
%\addcontentsline{toc}{chapter}{\listtablename}
%\listoftables
%\cleardoublepage

%	Listingsverzeichnis / Quelltextverzeichnis
%\lstlistoflistings
%\cleardoublepage

% Algorithmenverzeichnis
%\listofalgorithms
%\cleardoublepage

% Abkürzungsverzeichnis
% @stud: acronyms.tex bearbeiten
%% !TEX root =  master.tex
\clearpage
\chapter*{Abkürzungsverzeichnis}	
\addcontentsline{toc}{chapter}{Abkürzungsverzeichnis}

\begin{acronym}[XXXXXXX]
	\acro{Schnelltest}[Schnelltest]{Rapid Antigen whatever... TODO}

\end{acronym} 
%\cleardoublepage

%	Kurzfassung / Abstract
% @stud: abstract.tex bearbeiten
% !TEX root =  master.tex
\chapter*{Kurzfassung (Abstract)}
\addcontentsline{toc}{chapter}{Kurzfassung (Abstract)}

\subsubsection{Deutsch}
Die vorliegende Arbeit hat zum Ziel, die Effizienz der PCR-Analyse durch Pooling der Proben zu erhöhen.
Beantwortet werden soll die Frage, ob sich die Effizienz des PCR-Verfahren durch Pooling stark genug steigern lässt, um eine wirtschaftliche Alternative für das Einsatzgebiet von Schnelltests zu bieten?
Weiter wird untersucht, wie eine PCR-basierte Teststrategie im betrieblichen Umfeld realisiert werden könnte.

Erläutert wurden zunächst Methoden für die Testung und Grundlegende Konzepte des Poolings.
Es wurde die Effizienzsteigerungspotenzial durch Pooling aufgezeigt und errechnet.
Auf dieser Basis wurden PrePrints mit weitergehenden Poolingkonzepten analysiert, um zu prüfen ob diese weiteres Steigerungspotenzial bieten.

Ergebnis...

\subsubsection{Englisch} %DEEPL.com
The aim of this research paper is to increase the efficiency of PCR analysis by pooling samples.
The question to be answered is whether the efficiency of the PCR process can be increased strongly enough by pooling to offer an economical alternative for the application of rapid tests?
Further, it will be investigated how a PCR-based testing strategy could be realized in an operational environment.

Methods for testing and basic concepts of pooling were first explained.
The potential for efficiency gains through pooling was shown and calculated.
On this basis, PrePrints with more advanced pooling concepts were analyzed to see if they offer further potential for increase. 
\cleardoublepage

%%%%%%%%%%%%%%%%%%%%%%%%%%%%%%%%%%%%%%%%%%%%%%%%%%%%%%%%%%%%%%%%%%%%%%%%%%%%%%%%%%%%%%%%%%
% KAPITEL UND ANHÄNGE
%
% @stud:
%   - nicht benötigte: auskommentieren/löschen
%   - neue: bei Bedarf hinzufügen mittels input-Kommando an entsprechender Stelle einfügen
%%%%%%%%%%%%%%%%%%%%%%%%%%%%%%%%%%%%%%%%%%%%%%%%%%%%%%%%%%%%%%%%%%%%%%%%%%%%%%%%%%%%%%%%%%

\initializeText
\onehalfspacing

%%%%%%%%%%%%%%%%%%%%%%%%%%%%%%%%%%%
% KAPITEL
%
% @stud: einzelne Kapitel bearbeiten und eigene Kapitel hier einfügen
%

% Research Proposal einfügen
%% !TEX root =  master.tex

\chapter{Reserch Proposal vom 30.11.2021}
\section{Problemstellung und Forschungsziel}
Der Testung der Bevölkerung ist eine der wichtigsten Maßnahmen zur Pandemiebekämpfung.
Zielführend wäre deshalb, hierbei die ungenauen Antigen-Schnelltests\footnote{WuerzburgStudie}
durch das präziseren PCR-Verfahren\footnote{Polymerase Chain Reaction}
zu ersetzen.
Aktuell ist dies aufgrund der hierfür erforderlichen Laborkapazitäten und hohen Kosten nicht umsetzbar.\footnote{rki-bericht 2021}

Die vorgesehene Arbeit hat zum Ziel, existierende Pooling-Verfahren zu überprüfen.
Die Proben mehrerer Patienten werden hierbei kombiniert und gemeinsam getestet.
Diese Verfahren bieten erhebliche Steigerungspotenziale für die Kapazitäten der PCR-Testungen.\footnote{Aertzeblatt}
Bei der anlasslosen Massentestung könnte hierdurch eine Qualitätssteigerung erreicht werden.
Identifiziert werden soll ein Pooling-Verfahren, welches diese Steigerung ohne signifikanten Qualitätsverlust ermöglicht.
Ziel der Arbeit ist deshalb, Pooling-Methoden nach den folgenden Anforderungen zu prüfen:
\begin{itemize}
	\setlength{\itemsep}{-8pt}
	\item Potenzial zur Erhöhung der Testkapazitäten
	\item Robustheit gegen Fehlanwendung
	\item Risiko von falschen Ergebnissen
	\item Anwendbarkeit im betrieblichen Umfeld
\end{itemize}

Während der Pandemie haben sich viele Forschungsgruppen an Pooling-Verfahren gearbeitet.\footnote{viehweger increased 2020}
Hierbei sind viele unterschiedliche Ansätze entstanden.\footnote{verwilt evaluation 2021}
Die vorliegende Arbeit möchte diese Methoden zunächst analysieren und zu Clustern ähnlicher Verfahren aggregieren.
Anschließend soll  eine Überprüfung auf Effizienz und Robustheit der unterschiedlichen Verfahren stattfinden.
Hierfür sollen eine Simulation oder eine argumentativ-deduktive Analyse eingesetzt werden.

\textbf{Primäres Forschungsziel}\newline
Analyse existierender PCR-Pooling-Verfahren.\newline
Überprüfung dieser Methoden auf Effizienz und Robustheit.

\textbf{Sekundäres Forschungsziel}\newline 
Erarbeitung einer Referenzimplementierung für das betriebliche Umfeld.\newline

\section{Aktueller Stand der Wissenschaft}
\subsection{Das PCR-Testverfahren}
Um die Grundlagen für spätere Kapitel zu legen, soll zunächst die allgemeine Funktionsweise des PCR-Verfahrens erläutert werden.
Die verfahrensüblichen Qualiätsmerkmale Sensitivität und Spezifität werden beschrieben und auf Ablauf sowie Logistik der Testungen eingegangen.

Das PCR-Verfahren existierte bereits vor der COVID-19-Pandemie.
Es wird seit 1983
\footnote{wink pcr 1994}
zur Erkennung von Viruserkrankungen eingesetzt.
Sowohl wissenschaftliche Literatur\footnote{schmidt novel 2020}
als auch praxisnahe Publikationen\footnote{wehrle weber pcr 1994}
sind verfügbar.\footnote{clewley polymerase 1995}
Forschungsrelevante Aussagen zur Wirksamkeit und Fehlerquote der Testung werden hierbei ausschließlich auf Quellen gestützt, welche die wissenschaftlichen Qualitätsstandards erfüllen.
Für betriebswirtschaftliche und ablauforganisatorische Themenbereiche wird die Einbeziehung von praxisnahe Literatur als sinnvoll erachtet.

\subsection{PCR-Pooling-Verfahren}
PCR-Pooling-Verfahren wurden bereits vor der Pandemie zur Diagnose anderen Viruserkrankungen erfolgreich eingesetzt.
\footnote{Aertzeblatt}
Hierbei werden die Proben mehrerer Patienten vermischt, um durch einen gemeinsamen Test den Aufwand zu senken.
Im Laufe der Pandemie wurden von vielen Forschungsgruppen und Laboren Methoden entwickelt, um PCR-Pooling durchzuführen.
\footnote{calabrese how 2021}
Zur Robustheit des PCR-Verfahren gegen Verwässerung der Proben und den Skalierungsmöglichkeiten gibt es widersprüchliche Aussagen.
Einige Methoden empfehlen, dass maximal fünf Personen gemeinsam getestet werden.
\footnote{schmidt novel 2020}
Andere vermischen die Proben von 25-40 Patienten.
\footnote{verwilt evaluation 2021}
Das Ärzteblatt bescheinigt den Blutspendediensten die meiste Erfahrung mit PCR-Pooling-Verfahren.
\footnote{Aertzeblatt}
Um Blutspenden auf HIV und Hepatitis zu testen, kommen Pooling-Verfahren hier seit Jahrzehnten zum Einsatz.

Ein Vergleich dieser Pooling-Studien und der erforschten Methoden wird ein Schwerpunkt dieses Kapitels.
Ein Fokus ist hierbei die Fehleranfälligkeit der Tests bei unterschiedlichen Bedingungen und Verfahren.

Die wissenschaftliche Artikel, welche den Verfahren zugrunde liegen, werden aufgrund der Tagesaktualität teilweise von Pre-Print-Servern stammen.\footnote{viehweger increased 2020}
In diesen Fällen ist noch keinen Peer-Review erfolgt, weswegen diese Quellen besonders kritisch reflektiert und mit anderen Publikationen verglichen werden müssen.\footnote{verwilt evaluation 2021}
Die Einhaltung des wissenschaftlichen Anspruchs wird durch den Vergleich der vielen Publikationen und auf Basis der zugrundeliegenden Literatur sichergestellt.
Eine Validierung und Plausibilisierung der Methoden ist ohnehin im Rahmen der primären Forschungsfrage vorgesehen.

\subsection{Methoden der Kanalcodierung}
Ein Forschungsgebiet der Informatik ist die Integritätsprüfung von Speichern und Signalübertragungen.
Die Forscher entwickeln Algorithmen, für die Erkennung und Berichtigung von Fehlern.\footnote{hamming information 1987}
Die Anforderungen sind hierbei stark abhängig vom Anwendungsfall und der zu erwartenden Fehlerverteilung.\footnote{blahut algebraic 1992}

Dieses Kapitel dient dazu, ein Verständnis für die Funktionsweise unterschiedlicher Codierungsverfahren zu vermitteln.
Anhand von Beispielen werden die Unterschiede und Eigenschaften verschiedener Verfahren aufgezeigt.
Verdeutlicht wird hierdurch, nach welchen Kriterien Codierungsverfahren bewertet und verglichen werden können.
Aus der Kanalcodierung werden Anforderungen, Konzepte und Terminologie\footnote{dankmeier codierung 1994}
für die Analyse der Pooling-Verfahren übernommen.

Die Existenz dieses Kapitels ist noch ungewiss.
Sein Zweck wäre, die theoretische Grundlage für eine argumentativ-deduktive Analyse der Pooling-Methoden zu liefern.
Sollte diese Analyse zugunsten einer Simulation entfallen, ist dieses Kapitel obsolet.\footnote{Vgl. Gliederung der Arbeit}

\section{Forschungsmethodik und Vorgehen}
\subsection{Validierung der Modelle}
Im vorherigen Kapitel wurden anhand von internationalen Studien Verfahren für das PCR-Pooling erarbeitet.
Die ermittelten Modelle sollen in diesem Kapitel durch Forschungsmethoden der Wirtschaftsinformatik validiert werden.

Bei mangelhafter Umsetzung einer Pooling-Methode, könnten die Fehlerrate der Testverfahren steigen oder Proben kontaminiert werden.
Das Ergebnis wären ein Mehraufwand durch erneute Testung oder unentdeckte Fehldiagnosen.
Ziel ist es, eine geeignete Methode und Skalierung zu finden und diese auf Robustheit gegen Fehlern zu überprüfen.
Als Forschungsmethoden für die Validierung sind eine \textbf{Simulation} oder eine \textbf{argumentativ-deduktiven Analyse} vorgesehen.
Hierbei sollen die Grenzen der beschriebenen Verfahren erforscht werden.

Getestet werden sollen beispielsweise:
\begin{itemize}
	\setlength{\itemsep}{-8pt}
	\item Unterschiedliche Infektionswahrscheinlichkeiten in der Testgruppe
	\item Clusterbildung unter den Positivfällen
	\item Falschergebnis einzelner (Teil-)Testungen
	\item Kontaminierung der Proben
	\item Fehler bei der Probenvermischung
\end{itemize}

Aus den verfügbaren Pooling-Verfahren des vorherigen Kapitels soll so ein möglichst effizientes und robustes Modell gewählt werden.
Es sollen optimale Parameter für das Modell gefunden werden, um die in der Forschungsfrage formulierten Ziele zu erfüllen.

Die Ergebnisse der Validierung werden gegebenenfalls als Anpassungen in die Modellen eingearbeitet.
Dies soll die Grundlage für die Auswahl einer effizienten Methode sein, welche im Zuge der sekundären Forschungsfrage implementiert wird.

\subsection{Implementierung im betrieblichen Umfeld}
In diesem Kapitel soll eine Referenzimplementierung der erarbeiteten Modelle in das betriebliche Umfeld erstellt werden.
Hierbei handelt es sich um die sekundäre Forschungsfrage.
Die primären Forschung soll im Falle eines Ressourcenkonflikts priorisiert werden.\footnote{Vgl. Gliederung der Arbeit}

In der Medizin sowie im betrieblichen Umfeld, funktioniert Skalierung grundsätzlich anders als in der Informatik.
In der betrieblichen Umsetzung bedeutet die Verdopplung der Personenzahl einen massiven Mehraufwand bei Logistik und Organisation.
Um die ermittelte Effizienzsteigerung in der Praxis zu erreichen, müssen die Abläufe fehlerfrei ausgeführt werden.
Eine Aufgabe der Implementierung ist es, Risiken für operative Fehler minimieren.
Erreicht werden kann dies durch betriebliche Abläufe, Dokumentation und die Reduzierung der Arbeitsschritte.
Hierfür sollen praktische Empfehlungen gegeben werden.
Zudem wird die Logistik und entstehende Kosten der Testung beschrieben.

Dieses Kapitel wird sich auf einige Standardliteratur aus den Bereichen Prozess- und Ablauforganisation stützen.
Im Schwerpunkt handelt es sich hierbei allerdings um ein induktives Kapitel mit dem Ziel, einen Ausblick auf mögliche Implementierungsstrategien zu geben.
Eine abschließende Behandlung der betrieblichen Abläufe wird im Rahmen der Forschungsarbeit nicht möglich sein.
Aufgrund der Individualität jedes Unternehmens sollen allgemeingültige Empfehlungen gegeben werden.

\subsection{Zu erwartende Eigenbeiträge}
Die wissenschaftliche Lage verändert sich im Rahmen der Corona-Pandemie nahezu täglich und es ist schwierig, hierzu einen nachhaltigen Beitrag zu leisten.
Die vorgeschlagene Arbeit orientiert sich deshalb zwar am aktuellen Bedarf der pandemischen Lage - beschränkt sich jedoch nicht auf diesen.

Weder das PCR-Testverfahren noch die Idee zum Pooling von Proben ist neu in der COVID-19-Pandemie entstanden.
Diese Verfahren sind bewährt und werden auch nach der Pandemie noch zum Einsatz kommen.
Die Forschung nutzt die Pandemie somit als Anhaltspunkt und Praxisbeispiel, in der Hoffnung einen kurzfristigen Beitrag leisten zu können.
Die Ergebnisse sollen hierbei abstrahierbar für weitere Anwendungsfälle bleiben.

\section{Gliederung der Arbeit}
%\begin{figure}[h]
%	\centering
	%\includegraphics[height=.15\textwidth]{images/Untitled Diagram.drawio}
	%\caption{Geplanter Aufbau der Arbeit\footnotemark}
%\end{figure}

\paragraph{Einleitung}
Die Einleitung beginnt mit einer Beschreibung der Problemstellung und Zielsetzung der Arbeit.
Hierauf folgen die Forschungsfragen, welche sich am Research Proposal orientieren.
Am Ende des Einleitungskapitels wird das PCR-Verfahren im Allgemeinen vorgestellt, um eine Grundlage für die Bearbeitung der Forschungsfragen zu schaffen.

\paragraph{Primäre Forschungsfrage}
Die Beantwortung der primären Forschungsfrage beginnt mit einer Aufbereitung der bisherigen Forschung zu PCR-Pooling-Verfahren.
Für die Validierung sind zwei Forschungsansätze denkbar:
\begin{itemize}
	\setlength{\itemsep}{-8pt}
	\item Qualitativer Ansatz:
	Die Disziplin der Kanalcodierung wird vorgestellt und auf Basis dieses wissenschaftlichen Frameworks die Pooling-Verfahren formalisiert.
	Hierauf erfolgt eine argumentativ-deduktive Analyse, welche die Methode durch theoretische und qualitative Ansätze überprüft.
	\item Quantitativer Ansatz:
	Die Pooling-Verfahren werden in Software nachgebaut und quantitativ anhand einer Simulation analysiert.
	Es werden unterschiedliche Grenzfälle getestet, um die Auswirkung auf das Verfahren zu beobachten.
\end{itemize}

\paragraph{Sekundäre Forschungsfrage und Ausblick}
Die sekundäre Forschungsfrage ist die Implementierung der Methode im betrieblichen Umfeld.
Der Umfang dieser Forschung wird flexibel dem Ressourcenbedarf der primären Forschungsfrage angepasst.
Die Behandlung der Implementierung ist somit als eigenes Hauptkapitel denkbar.
Alternativ erfolgt eine Kürzung als Ausblick nach dem Ergebnis.

\paragraph{Ergebnis}
Die Arbeit endet mit einem Kapitel, in welchem die Erlebnisse zusammengefasst und Handlungsempfehlungen gegeben werden.
Es wird geprüft ob das Forschungsziel erreicht wurde und ob ein Optimierungspotenzial gegenüber den bisherigen Verfahren besteht.
Abhängig vom vorherigen Kapitel folgt ein Ausblick.

% Einleitung
% !TEX root =  master.tex
\chapter{Einleitung}
\section{Problemstellung}
Die Covid19-Pandemie existiert zum Zeitpunkt dieser Arbeit seit über 2 Jahren.
Mit unterschiedlichsten Maßnahmen wird versucht, die weitere Ausbreitung einzudämmen und die Kapazitäten des Gesundheitssystems nicht zu überlasten.
Neben Impfungen und Masken zählen auch Einschränkungen des öffenlichen Lebens und die Nachverfolgung von Infektionsketten zu den ergriffenen Maßnahmen.\footnote{(Quelle Verordnung)}

Ein elementarer Baustein der Pandemiestrategie ist zudem die Massentestung der Bevölkerung auf Infektion mit SARS-CoV2.
Hierbei erfolgt die anlasslose Massentestung üblicherweise mit Antigen-Schnelltests, während Verdachtsfälle über PCR-Tests überprüft werden.\footnote{(Quelle Verordnung)}
Unternehmen testen ihre Mitarbeitern, um Infektionen frühzeitig zu erkennen und eine Verbreitung zu vermeiden.
 Der Nachweis einer negativen Testung ist - alternativ zu einer Immunisierung - Voraussetzung für die Teilnahmen an vielen Bereichen des öffentlichen Lebens.\footnote{(Quelle Verordnung)}

Die Genauigkeit der Schnelltests ist allerdings oftmals nicht ausreichend.
Die Zulassungsstudien der Schnelltests zeigen sehr große Unterschiede in der Qualität.
(Hersteller) erkennt selbst bei geringer Viruslast XX Prozent der Infektionen
(Hersteller) dagegen zeigt selbst bei sehr hoher Virenlast nur XX Prozent der Infizierten richtig an.
\footnote{Zerforschung Jan 2022}

Im Frühjahr 2022 wurde die Kapazität für PCR-Testungen durch die stark gestiegenen Infektionszahlen überschritten.
Deshalb wurde die Möglichkeit für PCR-Testungen im Januar 2022 stark eingeschränkt.\footnote{Quelle neue Testverordnung}
Das qualitativ hochwertigere PCR-Verfahren steht seitdem nur noch zur Überprüfung eines positiven Antigen-Schnelltests zur Verfügung.
Diese erkennen allerdings wie oben ausgeführt nicht alle Infektionen, sodass viele Personen von einer Erkennung durch PCR ausgeschlossen sind.

\section{Zielsetzung und Forschungsfrage}

\textit{"ln most laboratories, the screening capacity is limited by the number of PCR reactions that can be performed in a day. It is, therefore, desirable to maximize the number of samples that can be tested per reaction."}\footnote{Viehweger Z21-23}

Im Laufe der Pandemie wurden von vielen Forschungsgruppen und Laboren Methoden entwickelt, um PCR-Pooling durchzuführen.
Diese machen es möglich, Patienten gemeinsam zu testen und Analysekapazitäten einzusparen.
Ziel der Arbeit ist es, die Kosten einer PCR-Testung zu senken und vorhandene Laborkapazitäten effizienter zu nutzen.
Hierfür soll das PCR-Poolingverfahren betrachtet werden.

%\subsubsection{Forschungsfragen}
Die Arbeit hat zum Ziel, die folgenden \textbf{Forschungsfragen} zu beantworten:

\begin{itemize}
	\item \textbf{Lässt sich die Effizienz des PCR-Verfahren durch Pooling stark genug steigern, um eine wirtschaftliche Alternative für das Einsatzgebiet von Schnelltests zu bieten?}
	\item \textbf{Wie könnte eine PCR-basierte Teststrategie im betrieblichen Umfeld realisiert werden?}
\end{itemize}

Die Hypothese der Arbeit ist, dass PCR durch Pooling effizienter sein kann als Schnelltests.
Hierfür soll überprüft werden, wie sich die Kosten des PCR-Verfahrens durch Pooling entwickeln.
Geprüft werden soll deshalb die Tauglichkeit für eine Massentestung.
Hierbei ist vor allem ein geringer Preis ausschlaggebend.

Um sich mit Schnelltests messen zu können, ist es notwendig die Prioritäten der PCR-Methode ähnlich festzulegen.
Für die Ziele dieser Arbeit ist es somit erforderlich, große Pools zu bilden um die notwendige Kostenreduzierung zu erreichen.
Die Präzision fällt hierbei allerdings unter die Schwelle dessen, was üblicherweise für PCR als akzeptabel betrachtet wird.
Durch die bereits diskutierte niedrige Erkennungsrate vieler Schnelltests, kann auch beim PCR-Pooling ein Verlust an Genauigkeit akzeptiert werden.

Diese Hypothese dieser Arbeit wird als bestätigt betrachtet, wenn das PCR-Pooling bei vergleichbaren Kosten wie ein Schnelltest eine höhere Erkennungsrate bietet.

\section{Abgrenzung}
\textbf{Vereinfachungen an den Modellen}
\begin{itemize}
	\item\textbf{Einsatzgebiet}
	Es wird ein Verfahren gesucht, welches als Ersatz für die anlasslose Massentestung geeignet ist.
	Es ist deswegen akzeptabel Parameter und Methoden zu betrachten, die unter einer hohen Prävalenz nicht tragbar sind.
	Die Überprüfung von Verdachtsfällen ist als Anwendungsfall zu trennen.
	Aus Gründen der Sicherheit und Geschwindigkeit sollten Personen mit Symoptomen durch klassische Einzeltests verifiziert werden.
	\item\textbf{Fehlerquote}\newline
	Der Umgang mit fehlerhaften Ergebnissen wird in Kapitel 3.2 theoretisch diskutiert.
	Mögliche Fehlerquoten fanden allerdings keinen Einzug in die Berechnungsmodelle für die Optimierung.
	Eine Berücksichtigung war aufgrund des Umfangs dieser Arbeit nicht möglich.
	Basierend auf den Empfehlungen in Kapitel 4.3 in Verbindung mit den Parametern welche in Kapitel 4.1 für die Optimierung verwendet werden, ist keine unüblich hohe Fehlerquote zu erwarten.
	\item\textbf{Potenzial durch mehrstufige Testung}
\end{itemize}







% mehrere Grundlagen- und Forschungs-Kapitel
% !TEX root =  master.tex
\chapter{Das PCR Verfahren}

\footnote{https://www.youtube.com/watch?v=FJFXYDP8N7M}
\footnote{https://factly.in/explainer-what-are-the-different-types-of-tests-being-used-in-india-for-covid-19-detection/}
\subsection{RT-PCR}
Durch Flüssigkeit werden Proteine und Fette gelöst / nur RNA bleibt übrig
4-5 Stunden bis ergebnis
90 proben können Zeitgleich getestet werden
Sensitivität: 60-90 Prozent
Spezifizität: 90-95 Prozent

\footnote{TrueNAT and CBNAAT}
60min bis ergebnis
Sens 80-80
Spez 90-95

\footnote{Rapid antigen}
15min
Probe ist nur 60min stabil
Sens 50-80
Spez 99-100

\footnote{igG Antigen Tests}
Testet auf Antikörper. Nicht für Diagnose.
Sens 92
Spez 97


\section{Funktionsweise des PCR-Verfahrens}
Ct werte und stuff

\cleardoublepage

\section{Möglichkeiten und Grenzen zum Pooling bei PCR}
\subsubsection{Unklare Ergebnisse und Nachtestung}
Durch Pooling besteht - abhängig vom Verfahren - das Risiko, dass die Ergebnisse nicht für alle Testpersonen eindeutig interpretiert werden können.
Bei vielen Pooling-Verfahren ergibt sich hierdurch die Notwendigkeit einer Nachtestung.
Ob dies der Fall ist und welche Quote der Testpersonen nachuntersucht werden muss, ist abhängig vom gewählten Verfahren.

Durch die erneute Testung geht Zeit verloren, bevor für alle Testpersonen das Ergebnis fest steht.
Die Proben müssen zudem ausreichend umfangreich sein, um genug Substanz für mehrere Testungen zu enthalten.
Beim Pooling des ersten Durchlaufs muss darauf geachtet werden, die Proben untereinander nicht zu kontaminieren.

\subsubsection{Prävalenz}
Inzidenz -> Prävalenz und erklären. 
\footnote{Beispiel Zeitungsbericht 70 Prozent Prävalenz}


\subsubsection{Verhinderung von Kontamination}
Die komplette Matrix sollte vor dem Pooling einmal dupliziert werden. 
Die für den aktuellen Test notwendigen Proben werden hierbei entnommen und im Duplikat gepoolt.
Für diesen Duplikationsschritt gibt es spezialisierte Laborgeräte, sodass dies in einem Arbeitsschritt für alle Proben durchgeführt werden kann.

%Doppelt
Hierfür muss zu beginn genug Probenmaterial bereit stehen und dieses darf nicht bei der Kombination kontaminiert werden.
Es empfiehlt sich, die Testmatrix zu beginn einmal zu klonen, um in der Originalmatrix ohne kontamination einzeln nachtesten zu können.

\subsubsection{Mögliche Poolgrößen und Erkennungsrate}
Um ein zuverlässiges Ergebnis zu liefern, dürfen die Proben nicht zu stark verwässert werden.
Hierbei wird empfohlen, maximal 20 Personen in einem Pool zu kombinieren. \footnote{Vieweger v1}
Die Testgruppe kann je nach Verfahren größer sein, solange kein Pool mehr als 20 Personen enthält.
Diese Poolgröße liegt laut Viehweger "comfortable above the detection rate"\footnote{Vieweger v1}

Eine höhere Verdünnung ist zulasten der Erkennungsrate problemlos möglich.
Abgewogen werden muss hierbei die Priorisierung zwischen Präzision und Kostenersparnis.

\subsubsection{Ziele dieser Arbeit}
Erklärtes ziel dieser Arbeit ist es, eine alternative zu Antigen-Schnelltest aufzuzeigen.
Die Rahmenbedingungen der bisherigen Testung sind somit
\begin{itemize}
	\item Sehr schnelle Anzeige des Ergebnisses
	\item Sehr geringe Kosten - unter 1€ Materialaufwand pro Stück
	\item Die Akzeptanz einer hohen Fehlerquote
	\item Massenscreenings mit sehr geringer Prävalenz
\end{itemize}
	
Um sich hiermit messen zu können, ist es notwendig die Prioritäten der PCR-Methode ähnlich festzulegen.
Einschränkungen der Präzision sind akzeptabel, um preislich mit den ungenauen Schnelltests zu konkurrieren.
Es ist außerdem eine für PCR-Verfahren unüblich niedrige Prävalenz zu erwarten, da normalerweise Verdachtsfälle mit PCR überprüft werden.

Für die Ziele dieser Arbeit ist es somit erforderlich, eine große Poolgröße zu wählen.
Diese ermöglicht eine Stärkere Kostenreduzierung.
Die Präzision fällt hierbei allerdings unter die Schwelle dessen, was üblicherweise für PCR als akzeptabel betrachtet wird.

\subsubsection{Politische Ebene}
Politisch ist anzumerken, dass bei den Corona-Schutzmaßnahmen zwischen präzissen PCR-Tests und ungenauen Schnelltests getrennt wird.
Die Teilnahme an einigen Veranstaltungen ist somit nur mit PCR-Test zulässig.
es kann unterstellt werden, dass der Gesetzgeber hierbei kein oder nur ein schwaches Pooling eingerechnet hat.
Sollte durch die gewählte Pooling-Methode die Genauigkeit deutliche reduziert sein, sollten deshalb nur Schnelltest-Bescheinigungen an die negativ getesteten Personen ausgestellt werden.


% !TEX root =  master.tex
\cleardoublepage
\chapter{Grundlagen des PCR-Poolings}
\section{Parameter und Kenngrößen für das Verfahren}
%Prävalenz
\begin{wrapfigure}{r}{0.44\textwidth}
	%\centering
	\includegraphics[width=.44\textwidth]{img/RKI_PCR_Positivrate}
	\caption{Prävalenz von PCR-Tests}
\end{wrapfigure}
Die \textbf{Prävalenz} ist die Quote, mit welcher eine Krankheit in einer Stichprobe vorkommt.\footnote{Leon Gordis S37}
Sie ist ähnlich der derzeit allgemein bekannteren Inzidenz, welche sich auf die Gesamtbevölkerung bezieht.
Bei einer anlasslosen, repräsentativen Testung der Bevölkerung kann die Prävalenz eines Tests gleich der Inzidenz sein.
Bei einer anlassbezogenen Testung werden allerdings meist deutlich höhere Prävalenzen beobachtet.
Gemäß aktuellem RKI-Wochenbereicht sind zwischenzeitlich über 40 Prozent der PCR-Tests positiv.
\footnote{RKI Wochenbericht}

%Mögliche Poolgrößen
Das PCR-Verfahren erlaubt grundsätzlich ein \textbf{Pooling} von mehreren Testpersonen.
Die Proben der Patienten werden hierbei zu einem Pool zusammengefasst und gemeinsam getestet.
Das PCR-Verfahren ist darauf ausgelegt, geringe DNA-Mengen zu einer nachweisbaren Menge zu vermehren.
Die Verwässerung der Probe ist bis zu einem gewissen Punkt deshalb unproblematisch für den Nachweis.
Auch eine (zu) hohe Verdünnung ist zulasten der Erkennungsrate problemlos möglich.
Abgewogen werden muss hierbei die Priorisierung zwischen Präzision und Kostenersparnis.
Eine Poolgröße von bis zu 20 Personen liegt laut Viehweger "comfortable above the detection rate"\footnote{Vieweger v1}
Andere Gruppe halten Poolgrößen von bis zu 90 Personen für akzeptabel.\footnote{Quelle 2 Pooling Verwässerung}
Es ist zu beachten, dass bei größeren Pools mehr Verdopplungsschritte notwendig sind, um dieselbe Virenmenge in der Probe zu erhalten.
Beim Pooling von 16 Personen liegt beispielsweise eine um $2^{4}$ niedrigere Virenlast vor.
Deshalb müssen 4 weitere Zyklen eingeplant werden.
\footnote{Vieweger v1}

\section{Umgang mit dem Ergebnis}
\begin{wrapfigure}{r}{0.4\textwidth}
	%\centering
	\includegraphics[width=.4\textwidth]{img/PoolAlleNegativ}
	\caption{Pooling benötigt für drei negative Personen nur ein Test}
\end{wrapfigure}
\textbf{Ergebnisinterpretation}
\begin{itemize}
	\item \textbf{Negatives Poolergebnis:}\newline
	Ein negatives Gesamtergebnis bedeutet, dass \textbf{jede Einzelprobe negativ} war.
	Es wurde somit durch einen Test festgestellt, dass alle Personen im Pool negativ sind.
		
	\item \textbf{Positives Poolergebnis:}\newline
	Ein positives Gesamtergebnis bedeutet, dass \textbf{mindestens eine Einzelprobe positiv} war.
	In diesem Fall müssen weitere Tests durchgeführt werden, um die positiven Einzelpersonen zu ermitteln.

\end{itemize}

\begin{wrapfigure}{r}{0.4\textwidth}
	%\centering
	\includegraphics[width=.4\textwidth]{img/PoolPositiv}
	\caption{Ein positiven Pool kann die positive Person nicht idendifizieren}
\end{wrapfigure}
\textbf{Unklare Ergebnisse}\newline
Durch Pooling besteht das Risiko, dass die Ergebnisse nicht für alle Testpersonen eindeutig interpretiert werden können.
Wie häufig dies der Fall ist und welcher Anteil der Testgruppe nachuntersucht werden muss, ist abhängig vom gewählten Verfahren.

Hierdurch werden \textbf{Nachtestungen} der betroffenen Personen notwendig.
Die Tests erfolgen hierbei nacheinander und sind statistisch unabhängig voneinander.
Manche Verfahren erfordern sogar mehrere sequenzielle Nachtestungen.

Um Nachtestungen zu ermöglichen, müssen die Proben ausreichend Substanz für mehrere Testungen enthalten.
Durch die erneute Testung verlängert sich der Zeitraum, bevor für alle Testpersonen das Ergebnis fest steht.
Dies kann abhängig von der Situation in welcher der Test benötigt wird nicht akzeptabel sein.

\cleardoublepage
\begin{wrapfigure}{r}{0.4\textwidth}
	%\centering
	\includegraphics[width=.4\textwidth]{img/Pipettenmatrix}
	\caption{Pipettenautomat}
\end{wrapfigure}
Beim ersten Poolingdurchlauf muss darauf geachtet werden, die Proben untereinander nicht zu kontaminieren.
Eine Verunreinigung der Originalproben würde eine spätere Nachtestung unmöglich machen.

Um eine \textbf{Kontamination} durch das Pooling zu verhindern, sollte die komplette Matrix vor dem Pooling einmal dupliziert werden. 
Die für den aktuellen Test notwendigen Proben werden hierbei entnommen und im Duplikat gepoolt.
Für diesen Duplikationsschritt gibt es spezialisierte Laborgeräte, sodass dies in einem Arbeitsschritt für alle Proben durchgeführt werden kann - teilweise sogar automatisiert.\footnote{https://www.genengnews.com/wp-content/uploads/2019/07/Eppendorf.jpg}

\begin{wrapfigure}{l}{0.4\textwidth}
	%\centering
	\includegraphics[width=.4\textwidth]{img/KomplexePools}
	\caption{Überlappende Pools}
	
	\includegraphics[width=.4\textwidth]{img/MehrerePositiv}
	\caption{Mehrere Positivfälle}
\end{wrapfigure}

\textbf{Überlappende Pools}\newline
Um das Problem der Nachtestungen zu lösen, kann man mehrere überlappende Tests durchführen.
Aus der Kombination der Ergebnisse ist es theoretisch möglich, die infizierte Person zu triangulieren.

Schwierig wird es hierbei, wenn mehrere Personen innerhalb der Testgruppe positiv sind.
Die positiven Tests lassen sich dann nicht mehr exakt einer Person zuordnen.

Das Ergebnis in Abbildung X.X legt nahe, dass die mittlere Person infiziert ist.
Beide Pools fallen positiv aus und nur die mittlere Person ist Teil beider Testgruppen.
Für diese Person wäre es somit naheliegend ein falsch-positives Ergebnis mitzuteilen.

Die Kosten der Testung haben sich zudem für alle Testgruppen erhöht, da durch diese Strategie für die erste Testrunde bereits zwei Tests notwendig sind, um eine Testgruppe von drei Personen abzubilden.

\cleardoublepage
\begin{figure}[h]
	\centering
	\includegraphics[width=.8\textwidth]{img/GrossePooluebersicht}
	\caption{Überlappende Pools}
\end{figure}

Weitere Schwierigkeiten ergeben sich, wenn für Personen gemischte Ergebnisse vorliegen.
Bei den drei Personen der rechten Testgruppe sind als beide Pooltests positiv ausgefallen.
In dieser Gruppe sind höchstwahrscheinlich eine oder mehrere Personen infiziert und die Gruppe muss nachgetestet werden.

Bei den Personen der linken Gruppe liegt allerdings ein positives und ein negatives Ergebnis vor.
Theoretisch kann durch das negative Ergebnis ausgeschlossen werden, dass eine Person dieser Gruppe infiziert ist.
In der Praxis ist allerdings kein Test 100 Prozent zuverlässig.
Für diese Personen liegt somit ein positiver Pool vor und das Negativergebnis könnte fehlerhaft sein.
Testet man nun zur Sicherheit nochmal alle?

Die \textbf{sicherste Variante} wäre, alle Personen nachzutesten die Teil eines positiven Pools waren.
Hierdurch wäre der Mehrwert durch das Pooling allerdings schnell verloren.
Abhängig vom Anwendungsfall kann es dagegen akzeptabel sein, einige Infektionen nicht zu erkennen.
Dies ist bei anlasslosen Massentestungen der Fall sein in denen kein Negativzertifikat ausgestellt wird.
Die Kostenoptimierung steht hierbei im Vordergrund, sodass \textbf{einige falsch-negative Ergebnisse akzeptabel} sein können. 

In der Praxis sollte meist ein Mittelweg gewählt werden.
Dieser könnte beispielsweise sein, alle mutmaßlich negativen Personen einer Testgruppe in einem gemeinsamen Pool nachzutesten.
Dieser Pool hat damit eine erwartete Prävalenz von null und sollte immer negativ ausfallen.
Hierdurch kann ermittelt werden, ob beim ersten Durchlauf Fehler passiert sind und Personen übersehen wurden.
Sollte dieser Pool positiv werden, müssen alle Teilnehmer einzeln nachgetestet werden.
Hierduch werden bereits drei sequenzielle Durchläufe notwendig
Die Übermittlung des Testergebnisses wird hierdurch stark verzögert, was ebenfalls Probleme verursachen kann.

\section{Ermittlung des Erwartungswertes}
\begin{wrapfigure}{r}{0.4\textwidth}
	%\centering
	\includegraphics[width=.4\textwidth]{img/EffizienzNegativ}
	\caption{Effizienz eines \newline negativen Pools}
	
	\includegraphics[width=.4\textwidth]{img/EffizienzPositiv}
	\caption{Effizienz eines \newline positiven Pools}
\end{wrapfigure}

Als \textbf{Effizienz} einer Poolingmethode wird nachfolgend der Multiplikator bezeichnet, welcher gegenüber Einzeltestungen erzielt werden kann.
Diese ist Abhängig von der Größe der Testgruppe und der Anzahl der Tests die erforderlich sind um den Infektionsstatus jeder Person zu klassifizieren.
Die Effizienz lässt sich somit beschreiben als $\frac{Anzahl Testpersonen}{Anzahl Tests} $.

Im bestmöglichen Fall ist die gesamte Testgruppe nicht infiziert.
Hierdurch fallen im ersten Durchlauf alle Tests negativ aus und die gesamte Testgruppe kann als negativ markiert werden.
Im Beispiel der Abbildung X.X ergibt sich eine Effizienz von 3.

Im Falle einer Nachtestung wird ein initialer Test für den Pool benötigt, welcher positiv ausfällt.
Danach werden nochmal Tests für jede Einzelperson benötigt.
Die Effizienz lässt sich somit beschreiben als $\frac{Anzahl Testpersonen (N)}{1 Pooltest + N Einzeltests} $.
Für den Positivfall liegt die Effizienz also bei 0,75.
Das Ergebnis ist damit schlechter als wenn direkt einzeln getestet worden wäre, da zuerst ein zusätzlicher Pooltest verwendet wurde.

Für den Vergleich von Teststrategien ist der \textbf{Erwartungswert für die Effizienz} relevant.
Die benötigte Anzahl der Tests ist abhängig vom Ergebnis der Pooltests.
Wenn alle Ergebnisse negativ sind, ist die Effizienz 3. Wenn das Poolergebnis positiv ist, liegt die Effizienz bei nur 0,75.
Der Erwartungswert ergibt sich aus diesen beiden Szenarien, gewichtet nach ihrer Eintrittswahrscheinlichkeit.
Die Wahrscheinlichkeit, dass jemand innerhalb der Testgruppe Infiziert ist, hängt von der Prävalenz und der Größe der Testgruppe ab.

Bei einer Prävalenz von 10 Prozent und drei statistisch unabhängigen Testpersonen, liegt die Wahrscheinlichkeit für einen Positivfall bei 30 Prozent. \textbf{Statistik prüfen}
Die Effizienz liegt also mit einer Wahrscheinlichkeit von 30 Prozent bei 0,75 und mit einer Wahrscheinlichkeit von 70 Prozent bei 3,0.
Daraus ergibt sich unter diesen Parametern eine gewichtete Effizienz für das Verfahren von 2,325.

\cleardoublepage

% !TEX root =  master.tex
\chapter{Optimierung und Potenziale}
\section{Optimierung der Poolgröße}
Im vorherigen Kapitel wurde aufgezeigt, dass die Effizienz einer Poolingmethode von zwei Parametern abhängt.
In Kombination ergeben diese beiden Werte die Wahrscheinlichkeit, mit welcher infizierte Personen innerhalb der Testgruppe sind.
\begin{itemize}
	\item \textbf{Prävalenz} Hieraus ergibt sich die Wahrscheinlichkeit, mit der Personen infiziert sind.
	Bei einem realen Testverfahren ist die Prävalenz ein externer Faktor.
	Sie ist Abhängig von der aktuellen Inzidenz und dem Umfeld der Testung.
	\item \textbf{Größe der Testgruppe} Diese kann vom Labor frei gewählt werden.
\end{itemize}

Dieser Zusammenhang ergibt die Funktion für den Erwartungswert:\newline
$\frac{Personenzahl}{(1 - (1-Prävalenz)^{Personenzahl} \cdot (Personenzahl + 1)) + ((1-Prävalenz)^{Personenzahl}) \cdot 1)}$

Für jede gegebene Prävalenz, lassen sich durch Auswahl der Personenanzahl unterschiedliche Verläufe der Effizienzkurve erreichen.
Die Anzahl der Personen pro Pool sollte deshalb anhand der Prävalenz gewählt werden, um den Erwartungswert zu maximieren.

Der Erwartungswert(Personenzahl, Prävalenz) lässt sich auch als Graph darstellen.
\begin{wrapfigure}{r}{0.48\textwidth}
	%\centering
	\includegraphics[width=.48\textwidth]{img/PraevalenzZuTestgruppe}
	\caption{\mbox{Effizienz unterschiedlicher} \mbox{Poolgrößen} nach Prävalenz\footnotemark}
\end{wrapfigure}
\footnotetext{Eigene Darstellung}
Wenn man eine der Variablen als Rahmenbedingung festsetzt, lässt sich der andere Wert gemeinsam mit der Effizienz als Diagramm zeichnen.
in Abbildung X.X wird die Personenanzahl auf 4, 8 und 16 festgesetzt und der Erwartungswert in Abhängigkeit der Prävalenz dargestellt.
Zu erkennen ist, dass höhere Personenzahlen bei niedrigen Prävalenzen die Effizienz enorm steigern können.
Wenn die Prävalenz allerdings steigt, werden die großen Pools schnell anfällig für Nachtestungen und verlieren so überproportional an Effizienz.

\cleardoublepage

Alternativ zur Darstellung nach Poolgröße kann auch die Prävalenz als gegeben festgesetzt werden.
In Abbildung X.X wird hierfür jedes Prävalenzniveau als eigener Graph dargestellt.
Das Optimum für die aktuelle Prävalenz liegt hierbei immer am Hochpunkt.
Die Poolgröße und hieraus resultierende Effizienz kann direkt auf den Achsen abgelesen werden.
\begin{figure}[h]
	\centering
	\includegraphics[width=1\textwidth]{img/EffizienzTestgruppePfeile}
	\caption{Effizienz eines Pools mit drei Personen nach Prävalenz\footnotemark}
\end{figure}
\footnotetext{Eigene Darstellung}

\begin{wraptable}{r}{7.7cm}
	\begin{tabular}{|r|r|c|r|}
		\hline
		Inzidenz&Prävalenz&Personen&Effizienz\\
		\hline
		35 & 0,00035 & 54 & 26,9x\\
		\hline
		50 & 0,00050 & 45 & 22,5x \\
		\hline
		100 & 0,00100 & 32 & 15,9x \\
		\hline
		250 & 0,00250 & 21 & 10,1x \\
		\hline
		2.000 & 0,02 & 8 & 3,6x\\
		\hline
		10.000 & 0,10  & 4 & 1,7x\\
		\hline
		25.000 & 0,25 & 3 & 1,1x\\
		\hline
	\end{tabular}
	\caption{Effizienzsteigerungspotenzial\footnotemark}
\end{wraptable} 
\footnotetext{Eigene Darstellung, Eine vollständige Gegenüberstellung in tabellarischer Form ist im Anhang dargestellt.}
Die bisherige Inzidenz lag bei 100.
Das Pooling an Punkt A war mit 32 Personen effizient und erreichte einen Faktor von 15,93.
Wenn die Inzidenz sich auf 250 erhöht, sinkt die Effizienz bei unveränderten Poolgröße auf Punkt B. Die Effizienz ist nur noch 9,24.
Für das neue Inzidenzniveau bei 250 kann das Pooling optimiert werden, indem die Poolgröße auf 20 gesenkt wird.
Dies bewirkt eine Linksverschiebung entlang der 250-Linie zu Punkt C.
Hier kann eine Effizienz von 10,12x erwartet werden.

\cleardoublepage

\section{Ausblick: Komplexe Poolingverfahren}
Dieser Abschnitt widmet sich einem Ausblick auf komplexere Poolingverfahren, welche im Umfang dieser Arbeit nicht näher beleuchtet werden können.
Durch die zunehmende Komplexität ergeben sich Potenziale für weitere Effizienzsteigerungen.
Diese sind allerdings auch mit neuen Risiken verbunden.

\textbf{Mehrdimensionale Pools}\ haben das Ziel, durch Überlappung der Pools den Bedarf einer Nachtestung bei einzelnen Positivfällen zu minimieren.
Die Testpersonen werden in einer AxB-Matrix angeordnet.
Die Proben werden dann für jede Spalte und jede Reihe gepoolt.
Allgemein formuliert lässt sich sagen:
Testbedarf pro Person =
$\frac{A+B}{A\cdot B}$

\begin{wrapfigure}{r}{0.48\textwidth}
	%\centering
	\includegraphics[width=.48\textwidth]{img/2d_Pool_1Positiv}
	\caption{Zweidimensionaler Pool mit einer positiven Probe\footnotemark}
\end{wrapfigure}
\footnotetext{Eigene Darstellung}
Für eine Testgruppe von 25 Personen, welche in einer 5x5 Matrix angeordnet sind, werden somit 5+5 Tests benötigt.
Die Effizienz läge bei 2,5x wenn alle Personen negativ getestet werden.
Vergleichen mit dem eindimensionalen Poolingverfahren klingt das zunächst nicht nach sehr viel.
Allerdings ist dieses Verfahren robust gegen einzelne Positivfälle.
Dies kann bei hohen Prävalenzen einen Vorteil bietet, da nicht alle Testpersonen erneut getestet werden müssen.
Zwei Positivfälle lassen sich beispielsweise mit nur vier Nachtestungen auflösen.
Hieraus ergibt sich, dass 25 Personen mit 10+4 Tests aufgelöst wurden.
Das Verfahren behält also selbst bei zwei positiven Proben eine Effizienz von 1,79 und verspricht damit deutlich robuster gegen hohe Prävalenzen zu sein.

Viehweger beschreibt ein Verfahren, um bei einer Prävalenz von 2 Prozent noch eine Effizienzsteigerung um den Faktor 5x zu erreichen.\footnote{Viehweger Z14}
Das in dieser Arbeit beschriebene Verfahren kommt hier nur auf einen Erwartungswert von 3,6x.
Für die Überprüfung von Verdachtsfällen mit hohen Prävalenzen, sollte deshalb diese Verfahren geprüft werden.

\cleardoublepage
% !TEX root =  master.tex
\chapter{Implementierung in der betrieblichen Teststrategie}
\section{Ort der Testdurchführung}

\subsubsection{Pooling im Unternehmen}
Das Pooling wird bei diesem Ansatz von Mitarbeitern des Unternehmens durchgeführt.
Das Labor muss nicht einmal zwangsläufig wissen, dass Pooling durchgeführt wird.
\footnote{Abhängig vom Grad der Verwässerung sollte diese Information mitgeteilt werden, um die Anzahl der Zyklen zu erhöhen.}

Keine Verarbeitung der Proben im Unternehmen aufgrund 
\begin{itemize}
	\item Unsachgemäße Handhabung
	\subitem Risiko der Ansteckung
	\subitem Risiko von fehlerhafter Verarbeitung
	\subitem Risiko der Kontamination der Probe
	\item Effizenz
	\subitem Geeigente Geräte im Labor
	\subitem Höhere Geschwindigkeit
	\subitem Routine in der Anwendung
\end{itemize}

Von einer Probenverarbeitung im Unternehmen wird deshalb abgeraten.

\subsubsection{Pooling im Labor}
Beim Pooling im Labor werden im Unternehmen nur die Proben entnommen, beschriftet und an das Labor gesendet.
Hierdurch wird Arbeitsaufwand an das Labor verlagert und es wird ein Labor benötigt, welches das Pooling anbietet.

Der deutliche Vorteil ist hierbei, dass das Pooling von Medizinisch geschultem Personal mit angemessenen Werkzeugen durchgeführt wird.
Hierdurch ist von einer geringeren Fehlerquote, höherer Effizienz und einer Risikoreduktion im Umgang mit den möglicherweise kontaminierten Proben auszugehen.
Eine Durchführung des Poolings im Labor wird deshalb nach Möglichkeit empfohlen.

% Hier entfernen ans Ende der Pooling-Entwicklung?
Auf der Evaluierung der Poolingmethoden wurden mathematisch sinnvolle Verfahren für die jeweiligen Inzidenzstufen ermittelt.
Diese sollen nun um weitere Parameter erweitert werden, um ihre Tauglichkeit im betrieblichen Umfeld zu ermitteln.


\section{Skalierung im betrieblichen Umfeld}
Skalierung im Betrieblichen Umfeld funktioniert grundsätzlich anders als in der Informatik.
Während große Speicherblöcke den Paritätsbedarf senken, ergibt sich durch die Vergrößerung einer Testgruppe ein deutlicher Mehraufwand an Logistik und Organisation. 
Diese Aspekte sollen im vorliegenden Kapitel Beachtung finden.

Grundsätzlich gibt es zwei Ansätze, das Pooling zu organisieren, welche nachfolgend kurz beschrieben werden.


\subsubsection{Organisation im Unternehmen}
Mitarbeiter bekommen persistenten Voucher-Barcode auf dem Alle Daten und auch Abteilung / Kontaktpersonen hinterlegt sind
Ggf. Kontaktpersonen über Plattform oder auf Zettel mit Nr selbst angeben.

Tests werden an MA verteilt oder zentral im Gebäude entnommen.
Die Teströhren bekommen einen Barcode und gehen unverändert ins Labor.

Die Probenentnahme muss von einer geschulten Person beaufsichtigt werden, um Fehlanwendung und Missbrauch zu verhindert.
Hierfür gibt es in vielen Betrieben bereits Personal, welches nach \footnote{§XXXX}
für die Beaufsichtigung der 3G-Nachweise zugelassen ist.

Die Teströhrchen müssen bereits im Unternehmen beschriftet werden, um die Ergebnisse später zuzuordnen.
Hierbei bietet es sich an, eine nicht datenschutzrelevante Liste mit Personalnummern zu verwenden.
Ggf. kann in Büros die Telefondurchwahl als Testnummer genutzt werden.


Müll durch Einmaltests beachten. ggf Glasröhrchen für Proben und abkochen.

\subsubsection{Politische Ebene}
Politisch ist anzumerken, dass bei den Corona-Schutzmaßnahmen zwischen präzissen PCR-Tests und ungenauen Schnelltests getrennt wird.
Die Teilnahme an einigen Veranstaltungen ist somit nur mit PCR-Test zulässig.
es kann unterstellt werden, dass der Gesetzgeber hierbei kein oder nur ein schwaches Pooling eingerechnet hat.
Sollte durch die gewählte Pooling-Methode die Genauigkeit deutliche reduziert sein, sollten deshalb nur Schnelltest-Bescheinigungen an die negativ getesteten Personen ausgestellt werden.
\cleardoublepage




%#########################################################
%#########################################################
\if{false}
\section{Planung aus RP}
Die sekundäre Forschungsfrage ist die Implementierung der Methode im betrieblichen Umfeld.
Der Umfang dieser Forschung wird flexibel dem Ressourcenbedarf der primären Forschungsfrage angepasst.
Die Behandlung der Implementierung ist somit als eigenes Hauptkapitel denkbar.
Alternativ erfolgt eine Kürzung als Ausblick nach dem Ergebnis.

\section{Obsidian Sammlung}
\subsection{Implementierung}
Bei der Informatik, sinkt der Paritätsbedarf bei einer erhöhten Datenmenge.
Hier ist in der Theorie eine endlose Steigerung der Datenmenge effizient, bis zu einem Punkt an dem mehr Bitfehler wahrscheinlich sind als das System berichtigen kann.

In diesem Kapitel sollen betriebswirtschaftliche Aspekte Beachtung finden, insbesondere Skalierung und Logistik.

Es soll auf Basis der Kenntnisse aus dem Algorithmen-Kapitel mehrere Berechnungsmethoden und Skalierungsfaktoren erarbeitet werden.

Für den Abstrich könnten handelsübliche Schnelltests eingesetzt werden. Damit wird der Abstrich gemacht und der Pufferbehälter genutzt.

Mit 1-2 Tropfen wird ein Schnelltest gemacht. Ist dieser Positiv, wird ein voller PCR-test gemacht. Mit den Restlichen 5-7 Tropfen der Probe kann im Falle eines Negativergebnis in den PCR-Algorithmus eingestiegen werden.

Symptome| Schnelltest postitiv| PCR-Algorithmus
--------|----------|---------
Erste Vorprüfung| |
Wenn Negativ ->|Probenentnahme und Vorcheck|
-- | Wenn negativ -> | Probenversand Labor

Im Selbstversuch waren meist 7-10 Tropfen aus den Schnelltests heraus zu bekommen. 3-4 reichen für den eigentlichen Schnelltest.
\fi

% Fazit und Ausblick
% !TEX root =  master.tex
\chapter{Ergebnis}
\subsubsection{Erkenntnisse der Arbeit}
Zusammenfassend kann aus der Forschungsarbeit abgeleitet werden:

\begin{itemize}
	\item \textbf{Pooling} Das PCR-Verfahren kann genutzt werden, um mehrere Proben gemeinsam zu testen.
	\item \textbf{Einfache Poolingverfahren} Bereit durch einfache Poolingverfahren mit Nachtestung im Falle eines positiven Pools kann eine deutliche Effizienzsteigerung gegenüber der PCR-Einzeltestung erreicht werden.
	\item \textbf{Mehrdimensionale Poolingverfahren} Durch komplexere Analysen und mehrdimensionale Testgruppen kann eine / keine / Im Bereich von ... eine signifikante Effizienzsteigerung gegenüber einfachen Poolingverfahren erreicht werden.
	\item \textbf{Potenzial} Das Effizienzsteigerungspotenzial durch Pooling hängt direkt von der Prävalenz der Testgruppe ab. Das Potenzial ist bei geringer Prävalenz besonders hoch.
	\item \textbf{Testort} Vor Ort im Unternehmen sollte nur die Probenentnahme durchgeführt werden. Hintergrund hier sind die effizienteren Methoden und das geschulte Personal, welchen in Laboren zur Verfügung steht.
	\item \textbf{Verzögerung} PCR basierte Verfahren liefern zwangsläufig ein späteres Ergebnis als Schnelltests
	\item \textbf{Erkenntnis} 
	\item	CBNAAT
	\item Die Nachtestung kann durch mehrstufiges Pooling optimiert werden.
	

	
\end{itemize}
\cleardoublepage

\subsubsection{Fazit zur Forschungsfrage}

\if{false}
\subsection{Known Issues}
A) Probleme bei der wissenschaftlichkeit einiger Passagen.

B) Man bescheinigt Leuten ein PCR-Negativergebnis, für die 3/4 Tests positiv waren.
- Risiko / Fehlerquote
- Verunsicherung
- Nachprüfung und resultierende Kosten

C) Fehler von PCR-Tests / Akzeptable Quote
D) Mögliche Anwendungsfehler
\fi


%%%%%%%%%%%%%%%%%%%%%%%%%%%%%%%%%%%
%% !TEX root =  master.tex
\chapter{Quellenarbeit}

\section{Verwilt / prePrint / Vier Versionen}
https://www.medrxiv.org/content/10.1101/2020.07.17.20152702v4


\section{Calabrese / prePrint / Zwei Versionen}
https://www.medrxiv.org/content/10.1101/2020.12.21.20248431v2


\section{Viehweger / prePrint / Zwei Versionen}
https://www.medrxiv.org/content/10.1101/2020.04.16.20067603v2

%%%%%%%%%%%%%%%%%%%%%%%%%%%%%%%%%%%
% ANHÄNGE
%
% @stud: einzelne Anhänge bearbeiten und eigene Anhänge hier einfügen 
%        die nachfolgenden Zeilen deaktivieren, wenn keine Anhänge verwendet werden
%
%\initializeAppendix
%% !TEX root =  master.tex
\chapter*{Anhang 1: Optimale Poolgrößen}
\begin{figure}[h]
	\centering
	\includegraphics[width=.60\textwidth]{img/TabelleOptimum}
	\caption*{Anhang 1: Tabelle Optimale Testgrößen\footnotemark}
\end{figure}
\footnotetext{Eigene Darstellung}

%\input{appendix2}
%%%%%%%%%%%%%%%%%%%%%%%%%%%%%%%%%%%

\singlespacing

%%%%%%%%%%%%%%%%%%%%%%%%%%%%%%%%%%%
% LITERATURVERZEICHNIS
% @stud: Literaturverzeichnis in Datei bibliography.bib anpassen. 
%
% Alternative zu Verwendung von \initializeBibliography: Citavi ...
% (dann \initializeBibliography auskommentieren und eigenes LaTex Coding verwenden)
%
%\ihead{}
\printbibliography[title=\Literaturverzeichnis] 
\cleardoublepage
% Kopiert aus Config
%%%%%%%%%%%%%%%%%%%%%%%%%%%%%%%%%%%

%%%%%%%%%%%%%%%%%%%%%%%%%%%%%%%%%%%
% INDEX
% @stud: ggf. Index auskommentieren, wenn nicht benötigt
%
\addcontentsline{toc}{chapter}{Index}
\printindex

\end{document}
